%% LyX 2.0.0 created this file.  For more info, see http://www.lyx.org/.
%% Do not edit unless you really know what you are doing.
\documentclass[onecolumn,english]{IEEEtran}
\usepackage[T1]{fontenc}
\usepackage[latin9]{inputenc}
\usepackage{float}
\usepackage{amsthm}
\usepackage{amstext}
\usepackage{graphicx}

\makeatletter

%%%%%%%%%%%%%%%%%%%%%%%%%%%%%% LyX specific LaTeX commands.
%% A simple dot to overcome graphicx limitations
\newcommand{\lyxdot}{.}

\floatstyle{ruled}
\newfloat{algorithm}{tbp}{loa}
\providecommand{\algorithmname}{Algorithm}
\floatname{algorithm}{\protect\algorithmname}

%%%%%%%%%%%%%%%%%%%%%%%%%%%%%% Textclass specific LaTeX commands.
\theoremstyle{plain}
\newtheorem{thm}{\protect\theoremname}
\theoremstyle{plain}
\newtheorem{prop}[thm]{\protect\propositionname}

\makeatother

\usepackage{babel}
\providecommand{\propositionname}{Proposition}
\providecommand{\theoremname}{Theorem}

\begin{document}

\section{Robust PCA with known rank: a block coordinate descent approach}


\subsection{Equivalence formulation of Robust PCA with rank information}

We first shows that in the robust PCA framework, the same probability
guarantee will still hold when extra information is given. Then we
derive a block-coordinate descent formulation for the case when rank
information is known.
\begin{prop}
\label{prop:restriction prob}Let $M=L_{0}+S_{0}$ , $rank(L_{0})\le r$
satisfy the Robust PCA assumptions. Then with high probability, the
following problems are equivalent:

\begin{eqnarray}
J_{1} & =\min_{L,S} & ||L||_{*}+\lambda||S||_{1}\label{eq:general}\\
 & s.t. & M=L+S\nonumber 
\end{eqnarray}
\begin{eqnarray}
J_{2} & =\min_{L,S} & ||L||_{*}+\lambda||S||_{1}\label{eq:restricted}\\
 & s.t. & M=L+S\nonumber \\
 &  & \text{and }T(L,S)\text{ holds}\nonumber 
\end{eqnarray}


where $T(L,S)$ are conditions on $(L,S)$ that $(L_{0},S_{0})$ also
satisfies.\end{prop}
\begin{IEEEproof}
We use subscript to denote the optimizer for $J_{1}$and $J_{2}$respectively.
With high probability, $(L_{1},S_{1})=(L_{0},S_{0})$. Now, over the
event of $(L_{1},S_{1})=(L_{0},S_{0})$, $(L_{1},S_{1})$ is also
a feasible for (\ref{eq:restricted} ). Since $J_{1}\le J_{2}$ always,
now $(L_{1},S_{1})$ achieve the bound for (\ref{eq:restricted} ).
Note that it should also be unique because it would contradict with
the recovery of (\ref{eq:general}).
\end{IEEEproof}
Now we apply Proposition (\ref{prop:restriction prob}) to the case
when the rank information is known and derive a block coordinate descent
framework for future algorithm design.

\begin{eqnarray}
J_{3} & = & \min_{M=L+S,rank(L)\le r}||L||_{*}+\lambda||S||_{1}\nonumber \\
 & = & \min_{L,S,\mu_{i},p_{i}q_{i}}||L||_{*}+\lambda||S||_{1}\nonumber \\
 & s.t. & S=M-L,L=\sum_{i=1}^{r}\mu_{i}p_{i}q_{i}^{T}\nonumber \\
 &  & ||p_{i}||_{2}=1,||q_{i}||_{2}=1,\mu_{i}\ge0\nonumber \\
\nonumber \\
 & = & \min_{\mu_{i},p_{i}q_{i}}\sum_{i=1}^{r}\mu_{i}+\lambda||M-\sum_{i=1}^{r}\mu_{i}p_{i}q_{i}^{T}||_{1}\nonumber \\
 & s.t. & ||p_{i}||_{2}=1,||q_{i}||_{2}=1,\mu_{i}\ge0\nonumber \\
\nonumber \\
 & = & \min_{\mu_{i},p_{i}q_{i}}\sum_{i=1}^{r}|\mu_{i}-0|+\lambda||M-\sum_{i=1}^{r}\mu_{i}p_{i}q_{i}^{T}||_{1}\label{eq:rank form}\\
 & s.t. & ||p_{i}||_{2}=1,||q_{i}||_{2}=1,\mu_{i}\ge0\nonumber 
\end{eqnarray}


Note that this formulation allows us to optimize over $\mu_{i},p_{i}q_{i}$
sequentially and for each step of optimization, the problem is a weighted
median problem where efficient algorithm is known. And by the Proposition
(\ref{prop:restriction prob}), we know that the formulation of (\ref{eq:rank form})
can recover the orginal $(L_{0},S_{0})$ with high probability.


\subsection{Simplification using $L_{1}$ heuristic}


\subsubsection{Introduction}

Recall that the nuclear norm introduced in the PCP scheme is resulted
because we would like to extract the low rank component from gross
random noise. Nuclear norm is used because it is a heuristic for penalizing
high rank matrix. Now, we consider the case when we have extra information/guess
from the data set that we know precisely what the rank of the matrix
are. Therefore, it is natural to introduce the following heuristics. 

\begin{eqnarray}
E^{*} & = & \min_{\{p_{j}\}\{q_{j}\},1\le j\le r}||M-\sum_{j=1}^{r}p_{j}q_{j}^{T}||_{1}\label{heu}
\end{eqnarray}



\subsubsection{Performance guarantee for the simplified version}

Regarding this, we provide some performance guarantee for the case
when the noise is bounded. One is a result for deterministic case
and the other is for the random case. 
\begin{prop}
Let $M=S+\sum_{i=1}^{r}p_{i}q_{i}^{T}$, $\frac{2}{\epsilon}||S||_{1}\le||\sum_{i=1}^{r}p_{i}q_{i}^{T}||_{1}$,
then, for the recovered $(\hat{L},\hat{S})$ from (\ref{heu}) , it
will satisfy 
\begin{eqnarray*}
\frac{||\sum_{i=1}^{r}p_{i}q_{i}^{T}-\hat{L}||_{1}}{||\sum_{i=1}^{r}p_{i}q_{i}^{T}||_{1}} & \le & \epsilon
\end{eqnarray*}
\end{prop}
\begin{IEEEproof}
Assume not, then 
\begin{eqnarray*}
||S||_{1} & \ge & ||\sum_{i=1}^{r}p_{i}q_{i}^{T}+S-\hat{L}||_{1}\\
 & \ge & ||\sum_{i=1}^{r}p_{i}q_{i}^{T}-\hat{L}||_{1}-||S||_{1}\\
 & > & \epsilon||\sum_{i=1}^{r}p_{i}q_{i}^{T}||_{1}-||S||_{1}
\end{eqnarray*}

\end{IEEEproof}
This gives a contradition, which is, 
\begin{eqnarray*}
\frac{2}{\epsilon}||S||_{1} & > & ||\sum_{i=1}^{r}p_{i}q_{i}^{T}||
\end{eqnarray*}

\begin{prop}
Let $M=\sum_{i=1}^{r}p_{i}q_{i}^{T}+S$, where $S_{i,j}\sim Uniform(-x_{s},x_{s})$,
$(p_{i})_{j}\sim Uniform(-x_{p},x_{p})$, $(q_{i})_{j}\sim Uniform(-x_{q},x_{q})$
all of the random variables are independent. With $|S|=k$ such that
$\lim_{n\to\infty}\frac{k^{2}}{n}$, then we have, 
\begin{eqnarray*}
\lim_{n\to\infty}P(\frac{||\sum_{i=1}^{r}p_{i}q_{i}^{T}-\hat{L}||_{1}}{||\sum_{i=1}^{r}p_{i}q_{i}^{T}||_{1}}>\epsilon) & = & 0
\end{eqnarray*}
\end{prop}
\begin{IEEEproof}
Let $E$ be the error event that $\frac{||\sum_{i=1}^{r}p_{i}q_{i}^{T}-\hat{L}||_{1}}{||\sum_{i=1}^{r}p_{i}q_{i}^{T}||_{1}}>\epsilon$.
If error occurs, 
\begin{eqnarray*}
kx_{s} & \ge & ||\sum_{i=1}^{r}p_{i}q_{i}^{T}+S-\hat{L}||_{1}\\
 & \ge & ||\sum_{i=1}^{r}p_{i}q_{i}^{T}-\hat{L}||_{1}-||S||_{1}\\
 & \ge & \epsilon||\sum_{i=1}^{r}p_{i}q_{i}^{T}||_{1}-kx_{s}\\
 & \ge & \epsilon\sqrt{\sum_{l_{1}l_{2}}(\sum_{i=1}^{r}(p_{i})_{l_{1}}(q_{i})_{l_{2}})^{2}}-k_{s}
\end{eqnarray*}


Thus, 
\begin{eqnarray*}
 & Pr(E)\\
\le & Pr((\frac{2kx_{s}}{\epsilon})^{2}\ge\sum_{l_{1}l_{2}}(\sum_{i=1}^{r}(p_{i})_{l_{1}}(q_{i})_{l_{2}})^{2})\\
\le & Pr((\frac{2kx_{s}}{\epsilon})^{2}\ge\sum_{l_{1}=1}^{n}(\sum_{i=1}^{r}(p_{i})_{l_{1}}(q_{i})_{l_{1}})^{2})\\
= & Pr(\frac{1}{n}(\frac{2kx_{s}}{\epsilon})^{2}\ge\frac{1}{n}\sum_{l_{1}=1}^{n}(\sum_{i=1}^{r}(p_{i})_{l_{1}}(q_{i})_{l_{1}})^{2})
\end{eqnarray*}


Moreover, as $E(\sum_{i=1}^{r}(p_{i})_{l_{1}}(q_{i})_{l_{1}})^{2})=\frac{r}{3}x_{p}^{2}x_{q}^{2}$,
by law of large number, $\frac{1}{n}\sum_{l_{1}=1}^{n}(\sum_{i=1}^{r}(p_{i})_{l_{1}}(q_{i})_{l_{1}})^{2})\to\frac{r}{3}x_{p}^{2}x_{q}^{2}$.
Thus, since $\frac{1}{n}(\frac{2kx_{s}}{\epsilon})^{2}\to0$. This
gives $Pr(E)\to0$ as $n\to\infty$.
\end{IEEEproof}

\subsection{Algorithms derivation}

Note that the form for both (\ref{eq:rank form}) and (\ref{heu})
are similar. Indeed, one can generalize the method for $r=1$ in (\ref{heu})
to higher dimensions for both (\ref{eq:rank form}) and (\ref{heu}).
Therefore, we restrict our discussion to $r=1$ in (\ref{heu}) for
the following discussion.

Let $M=(a_{i,j})\in R^{mXn}$. We now employ the block-coordinate
descent approach to solve this problem. Note that 

\begin{eqnarray}
\min_{p} & ||M-pq^{T}||_{1} & =\sum_{i=1}^{m}\min_{t}(\sum_{j=1}^{n}|a_{i,j}-tq_{j}|)\\
 &  & =\sum_{i=1}^{m}\min_{t}(\sum_{j=1}^{n}|q_{j}||t-\frac{a_{i,j}}{q_{j}}|)\nonumber 
\end{eqnarray}
\begin{eqnarray}
\min_{q} & ||M-pq^{T}||_{1} & =\sum_{j=1}^{n}\min_{t}(\sum_{i=1}^{m}|a_{i,j}-tp_{i}|)\\
 &  & =\sum_{j=1}^{n}\min_{t}(\sum_{j=1}^{n}|p_{i}||t-\frac{a_{i,j}}{p_{i}}|)\nonumber 
\end{eqnarray}


And for solving the subproblem of finding 
\[
\min_{t}\sum_{k=1}^{k_{0}}c_{i}|t-d_{i}|
\]
where $c_{i}\ge0$ is basically finding the weighted median and can
be done by the following method with complexity $O(k_{0}\log k_{0})$
mostly on sorting the sequence. We call it WMH.

\begin{algorithm}[h]
\begin{enumerate}
\item We first sort $\vec{\ensuremath{d}}$ s.t. $d_{i_{1}}\le d_{i_{2}}\le...\le d_{i_{k_{0}}}$ 
\item We then find $k^{'}$s.t. 
\begin{eqnarray*}
\sum_{\theta=1}^{k^{'}-1}c_{i_{\theta}} & \le & \sum_{\theta=k^{'}}^{k_{0}}c_{i_{\theta}}\\
\sum_{i=1}^{k^{'}}c_{i_{\theta}} & \ge & \sum_{i=k^{'}+1}^{k_{0}}c_{i_{\theta}}
\end{eqnarray*}

\item We then set $t^{*}$to be $d_{i_{k^{'}}}$
\end{enumerate}
\caption{WMH $(k_{0},\vec{c},\vec{d})$}
\end{algorithm}


This algorithm is optimal in finding $t$. This is justified by using
the property of sub-differential of $||\cdot||_{1}$ and note that
$0\in\partial(\sum_{k=1}^{k_{0}}c_{i}|t^{*}-d_{i}|)$. 

Now we are ready to state the power iteration method to solve the
optimization rank-1 optimization problem. We call it Poweriteration.

\begin{algorithm}[h]
Repeat
\begin{enumerate}
\item $p_{i}\leftarrow wmh(n,abs(q),M(i,:)./q)$ for each i
\item $p\leftarrow\frac{p}{max(p)}$
\item $q_{j}\leftarrow wmh(n,abs(p),M(:,j)./p)$ for each j
\end{enumerate}
Until stopping criterion is met

\caption{Poweriteration($M$)}
\end{algorithm}



\subsection{Sensitivity to $\lambda$}

Note that in the robust PCA framework, the parameter $\lambda$ is
explicit stated as $\sqrt{\frac{1}{n}}$ and when the $\lambda$ is
too large or too small, it would significantly affect the recovery.
However, in the case that the rank information is explicitly stated.
The effect of $\lambda$ is different. As we have seen in previous
section, when the value of $\lambda$is large, it correspond to the
$L_{1}$ heuristic and there is some guarantee on recovery. Therefore,
it is natural to ask if we can have very small $\lambda$ and have
the robust PCA framework to hold. In particular, we specialize to
the rank 1 case, and it turns out that it cannot be done, as demonstrated
as follows.

Recall that if we directly apply robust PCA , we will get, 

\begin{eqnarray*}
 &  & \min_{M=L+S,rank(L)\le1}||L||_{*}+\lambda||S||_{1}\\
 & = & \min_{S=M-pq^{T},L=pq^{T}}||L||_{*}+\lambda||S||_{1}\\
 & = & \min_{S=M-pq^{T},L=pq^{T}}||pq^{T}||_{*}+\lambda||M-pq^{T}||_{1}\\
 & = & \min_{p,q:||p||_{2}=1}||pq^{T}||_{*}+\lambda||M-pq^{T}||_{1}\\
 & = & \min_{p,q:||p||_{2}=1}||q||_{2}+\lambda||M-pq^{T}||_{1}\\
 & = & \min_{p:||p||_{2}=1}\min_{q}||q||_{2}+\lambda||M-pq^{T}||_{1}
\end{eqnarray*}


Now, for every fixed $p$, we consider the subproblem, if we directly
apply the Robust PCA with $\lambda\le\frac{1}{n}$, 
\begin{eqnarray*}
 &  & \min_{q}||q||_{2}+\lambda||M-pq^{T}||_{1}\\
 & = & \min_{q}\max_{||u||_{2}\le1,||V||_{\infty\le1}}u^{T}q+\lambda Tr(V^{T}(M-pq^{T}))\\
 & = & \max_{||u||_{2}\le1,||V||_{\infty\le1}}\min_{q}u^{T}q+\lambda Tr(V^{T}(M-pq^{T}))\\
 & = & \max_{||u||_{2}\le1,||V||_{\infty\le1},u=\lambda V^{T}p}Tr(V^{T}M)\\
 & = & \max_{||\lambda V^{T}p||_{2}\le1,||V||_{\infty\le1}}Tr(V^{T}M)\\
 & = & \max_{||V^{T}p||_{2}\le n,||V||_{\infty\le1}}Tr(V^{T}M)
\end{eqnarray*}


Now note that, since $||p||_{2}\le1$, we have $||V^{T}p||_{2}\le\sqrt{\sum_{i=1}^{n}\sigma_{i}(V^{T}V)}=\sqrt{Tr(V^{T}V)}\le\sqrt{n^{2}||V||_{\infty}}$.
Thus, the optimal value is 
\begin{eqnarray*}
 &  & \min_{p,q:||p||_{2}=1}||pq^{T}||_{*}+\lambda||M-pq^{T}||_{1}\\
 & = & \min_{p:||p||_{2}=1}\max_{||V||_{\infty\le1}}Tr(V^{T}M)\\
 & = & \min_{p:||p||_{2}=1}||M||_{1}\\
 & = & ||M||_{1}
\end{eqnarray*}


And it is achieved by $pq^{T}=0$ matrix, which deviates from what
we expect to recover. 


\subsection{Simulation }


\subsubsection{Comparison between Robust PCA and $L_{1}$ heuristics}

We simulate the Robust PCA scheme and $L_{1}$ heuristics using the
power method. Note that in using the power iteration method for Robust
PCA, we would not update $\mu$ if the value of that iteration is
0 because this will make the algorithm to converge to the wrong value(as
observed from simulation, this happens quite frequently so this conditioning
is needed).

In the simulation, we use randomly generated the entries of $p$ and
$q$ as $N(0,1)$ iid. And then we randomly generate sparse matrix
with sparse support uniformly distributed across the nXn matrix. And
each sparse entry has a value with distribution of $N(0,1)$. We then
plot the graph of different degree of sparsity and the corresponding
effectiveness of the optimization heuristic in extracting the original
$pq^{T}$. The result is as follows.

\includegraphics[width=8cm]{\string"C:/Users/kk/Desktop/EE227A/project/matlab experiments/compare\string".jpg}
\end{document}

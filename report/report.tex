\documentclass{../common/projectreport}
\newcommand{\norm}[3]{{\left\|#1\right\|}_{#2}^{#3}}
\newcommand{\normball}[2]{{\mathbb{B}_{#1}^{#2}}}
\newcommand{\proximal}[2]{{\prox_{#1}(#2)}}
\newcommand{\forAll}{{\forall \,}}
\newcommand{\Rbb}{{\mathbb R}}
\newcommand{\Rbf}{{\mathbf R}}
\newcommand{\Lhat}{{\hat{L}}}
\newcommand{\Shat}{{\hat{S}}}
\newcommand{\Yhat}{{\hat{Y}}}
\newcommand{\Xhat}{{\hat{X}}}
\newcommand{\Nbb}{{\mathbb N}}
\newcommand{\Fbb}{{\mathbb F}}
\newcommand{\Cbb}{{\mathbb C}}
\newcommand{\Tbb}{{\mathbb T}}
\newcommand{\Zbb}{{\mathbb Z}}
\newcommand{\etabar}{{\bar{\eta}}}
\newcommand{\kbar}{{\bar{k}}}
\newcommand{\rbar}{{\bar{r}}}
\newcommand{\xibar}{{\bar{\xi}}}
\newcommand{\ubar}{{\bar{u}}}
\newcommand{\vbar}{{\bar{v}}}
\newcommand{\uhat}{{\hat{u}}}
\newcommand{\vbf}{{\mathbf{v}}}
\newcommand{\vhat}{{\hat{v}}}
\newcommand{\xbar}{{\bar{x}}}
\newcommand{\xhat}{{\hat{x}}}
\newcommand{\xdot}{{\dot{x}}}
\newcommand{\Xdot}{{\dot{X}}}
\newcommand{\xtilde}{{\tilde{x}}}
\newcommand{\xtildedot}{{\dot{\tilde{x}}}}
\newcommand{\Phitilde}{{\tilde{\Phi}}}
\newcommand{\phat}{{\hat{p}}}
\newcommand{\shat}{{\hat{s}}}
\newcommand{\xbf}{{\mathbf{x}}}
\newcommand{\wbf}{{\mathbf{w}}}
\newcommand{\Pbf}{{\mathbf{P}}}
\newcommand{\Xbf}{{\mathbf{X}}}
\newcommand{\Ebf}{{\mathbf{E}}}
\newcommand{\Obf}{{\mathbf{0}}}
\newcommand{\ebf}{{\mathbf{e}}}
\newcommand{\alphabf}{{\mathbf{\alpha}}} 
\newcommand{\alphabar}{{\bar{\alpha}}} 
\newcommand{\sgn}{\text{sgn}}
\newcommand{\Range}{\mathcal{R}}
\newcommand{\Nullspace}{\mathcal{N}}

\newcommand{\pderiv}[2]{\frac{\partial #1}{\partial #2}}
\newcommand{\pderivsq}[2]{\frac{\partial^2 #1}{\partial #2^2}}
\newcommand{\pderivxy}[3]{\frac{\partial^2 #1}{\partial #2 \partial #3}}

\newcommand{\One}{{\mathbf{1}}}

\newcommand{\Ccal}{{\mathcal{C}}}
\newcommand{\Dcal}{{\mathcal{D}}}
\newcommand{\Fcal}{{\mathcal{F}}}
\newcommand{\Hcal}{{\mathcal{H}}}
\newcommand{\Lcal}{{\mathcal{L}}}
\newcommand{\Ncal}{{\mathcal{N}}}
\newcommand{\Pcal}{{\mathcal{P}}}
\newcommand{\Qcal}{{\mathcal{Q}}}
\newcommand{\Rcal}{{\mathcal{R}}}
\newcommand{\Scal}{{\mathcal{S}}}
\newcommand{\Tcal}{{\mathcal{T}}}
\newcommand{\Ucal}{{\mathcal{U}}}
\newcommand{\Xcal}{{\mathcal{X}}}
\newcommand{\Ycal}{{\mathcal{Y}}}


\newcommand{\ifthen}[2]{\textbf{if} #1 \textbf{then} #2}
\newcommand{\ifthenreturn}[2]{\textbf{if} #1 \textbf{then return} #2}
\newcommand{\then}[1]{\textbf{then} #1}
\newcommand{\return}[1]{\textbf{return} #1}
\newcommand{\returns}[1]{\textbf{returns} #1}
\newcommand{\thenreturn}[1]{\textbf{then return} #1}
\newcommand{\foreachindo}[3]{\textbf{for each} #1 \textbf{in} #2 \textbf{do} #3}
\newcommand{\algindent}[1]{\STATE\hspace{#1\algorithmicindent}}

\DeclareMathOperator{\blkdiag}{blkdiag}
\DeclareMathOperator{\trace}{\mathbf{Tr}}
\DeclareMathOperator{\Convh}{Conv}
\DeclareMathOperator{\rank}{rk}
\DeclareMathOperator{\diag}{diag}
\DeclareMathOperator{\Realpart}{Re}
\DeclareMathOperator{\Imagpart}{Im}
\DeclareMathOperator{\Poles}{Poles}
\DeclareMathOperator{\spanspace}{span}
\DeclareMathOperator*{\argmax}{arg\,max}
\DeclareMathOperator*{\argmin}{arg\,min}
\DeclareMathOperator{\Onebf}{{\mathbf{1}}}
\DeclareMathOperator{\dom}{dom}
\DeclareMathOperator{\grad}{\nabla\!\!}
\DeclareMathOperator{\epi}{epi}
\DeclareMathOperator{\sign}{sign}
\DeclareMathOperator{\divergence}{div}
\DeclareMathOperator{\prox}{prox}

\newcommand\independent{\protect\mathpalette{\protect\independenT}{\perp}}
\def\independenT#1#2{\mathrel{\rlap{$#1#2$}\mkern2mu{#1#2}}}


\newcommand{\Acal}{{\mathcal{A}}}
\newcommand{\Abar}{{\bar{A}}}


\theoremstyle{plain}
\newtheorem{thm}{Theorem}
\theoremstyle{plain}
\newtheorem{fact}[thm]{Fact}
\theoremstyle{plain}
\newtheorem{prop}[thm]{Proposition}



%============================================================================================

\title{\Large Robust Principal Component Analysis}
\author{Maximilian Balandat \and Walid Krichene \and Chi Phang Lam \and Ka Kit Lam}

%============================================================================================
\begin{document}
\maketitle
Given a superposition $M = L_0 + S_0$ of a low-rank matrix $L_0$ and a sparse matrix $S_0$, Robust Principal Component Analysis \cite{Candes:2011fk} is the problem of recovering the low-rank and sparse components. Under suitable assumptions on the rank and incoherence of $L_0$, and the distribution of the support of $S_0$, the components can be recovered exactly with high probability, by solving the Principal Component Pursuit (PCP) problem given by

\begin{equation}
\begin{aligned}
&\text{minimize} && \|L\|_* + \lambda \|S\|_1 \\
&\text{subject to} && L+S = M
\label{PCP}
\end{aligned}
\end{equation}

Principal component pursuit minimizes a linear combination of the nuclear norm of a matrix $L$ and the $\ell_1$ norm of $M-L$. Minimizing the $\ell_1$ norm is known to favour sparsity, while minimizing the nuclear norm $\|L\|_* = \sum_{\sigma \in \sigma(L)} \sigma$ is known to favour low-rank matrices (intuitively, favours sparsity of the vector of singular values).

%============================================================================================
\section{Introduction}
Assume we are given a superposition of a low-rank matrix and a sparse matrix, $M \in \mathbb{R}^{n_1 \times n_2}$ given by
\[
M = L_0 + S_0
\]

One cannot expect to recover the components exactly in the most general case. Assume for example that $L_0$ is such that $(L_0)_{ij} = \delta_i^1\delta_j^1$, and $S_0 = -L_0$. Both matrices are sparse and low-rank, and clearly one cannot expect to recover the components in this case, since the observed matrix is $M = 0$. Therefore assumptions are made on the incoherence of $L_0$ and the support of $S_0$.

%-----------------------------------------------------------------------------------------------------------------------------------------------------------------
\subsection{Incoherence}
The Incoherence conditions describe how much the singular vectors of a given matrix are aligned with the vectors of the canonical basis.

Let the SVD of $L_0$ be given by

\begin{align}
L_0 = U\Sigma V^* = \sum_{i=1}^r \sigma_i u_i v_i^*
\end{align}

Then the incoherence conditions are given by

\begin{equation}
\begin{aligned}
\max_i \|U^*e_i\|_2^2 \leq \frac{\mu r}{n_1}, && \max_i \|V^*e_i\|_2^2 \leq \frac{\mu r}{n_2}
\label{incoherence1}
\end{aligned}
\end{equation}

and

\begin{equation}
\| U V^* \|_\infty \leq \sqrt{\frac{\mu r}{n_1 n_2}} \label{incoherence2}
\end{equation}

These conditions require the singular vectors to be ``spread'' enough with respect to the canonical basis. Intuitively, if the singular vectors of the low-rank matrix $L_0$ are aligned with a few canonical basis vectors, then $L_0$ will be sparse and hard to distinguish from the sparse corruption matrix $S_0$.


%============================================================================================
\section{Main Result}
\begin{theorem}
Suppose $L_0 \in \mathbb{R}^{n \times n}$ satisfies incoherence conditions~(\ref{incoherence1}) and (\ref{incoherence2}) and that the support of $S_0$ is uniformly distributed among all sets of cardinality $m$. Then $\exists c$ such that with high probability over the choice of support of $S_0$ (at least $1-cn^{-10}$), Principal Component Pursuit with $m = 1/\sqrt{n}$ is exact, i.e. $\hat{L} = L_0$ and $hat{S} = S_0$ provided that

\begin{equation}
\begin{aligned}
\text{rank}(L_0) \leq \frac{\rho_r}{\mu} \frac{n}{(\log n)^2} && \text{ and } && m \leq \rho_s n^2
\end{aligned}
\end{equation}

\end{theorem}

Above, $\rho_r$ and $\rho_s$ are positive numerical constants. Note in particular that no assumptions are made on the magnitudes of the nonzero entries of $S_0$.


%============================================================================================
\section{Proof}
%-----------------------------------------------------------------------------------------------------------------------------------------------------------------
\subsection{Prelimineries}
\begin{itemize}
\item The subgradient of of the $\ell_1$ norm at $S_0$ supported on $\Omega$ is of the form $\sgn(S_0)+F$ where $P_\Omega F = 0$ and $\|F\|_\infty \leq 1$.
\item The subgradient of the nuclear norm at $L_0 = U\Sigma V^*$ is of the form $U V^* + W$ where 
\begin{equation}
\begin{aligned}
&U^*W = 0 \\
& WV = 0\\
& \|W\| \leq 1
\end{aligned}
\end{equation}
or equivalently
\[
P_T W = 0
\]
where $T$ is the linear space of matrices defined by
\[
T = \{UX^* + YV^*, \ , X, Y \in \mathbb{R}^{n \times r}\}
\]
\end{itemize}





%-----------------------------------------------------------------------------------------------------------------------------------------------------------------
\subsection{Elimination Theorem}
The following elimination theorem states the intuitive fact that if PCP exactly recovers the components of $M = L+S$, then it also exactly recovers the components of $M = L+S'$ where $S'$ is a trimmed version of $S$ ($\text{supp}(S')\subset \text{supp}(S)$ and $S$ and $S'$ coincide on $\text{supp}(S')$)

\begin{theorem}
Suppose the solution to the PCP problem (\ref{PCP}) with input data $M_0 = L_0 + S_0$ is unique and exact, and consider $M_0' = L_0 + S_0'$ where $S_0'$ is a trimmed version of $S_0$. Then the solution to (\ref{PCP}) with input $M_0'$ is exact as well.
\end{theorem}


%-----------------------------------------------------------------------------------------------------------------------------------------------------------------
\subsection{Derandomization}
Derandomization is used to show equivalence between the problem where the signs of the entries of $S_0$ are random, and the problem where the entries of $S_0$ have fixed signs.

%-----------------------------------------------------------------------------------------------------------------------------------------------------------------
\subsection{Dual certificate}
The following lemma gives a simple sufficient condition for the pair $(L_0,S_0)$ to be the unique optimal solution to PCP.

\begin{lemma}
Assume that $\|P_\Omega P_T \| < 1$. Then $(L_0,S_0)$ is the unique solution to PCP if $\exists (W, F)$ such that
\begin{equation}
\begin{aligned}
& UV^* + W = \lambda(\text{sign}(S_0) + F) \\
& P_T W = 0 \\
& \|W\|<1 \\
& P_\Omega F = 0 \\
& \|F\|_\infty < 1
\end{aligned}
\end{equation}

\end{lemma}

\begin{proof}

\end{proof}


The proof will use a similar result, given by the following Lemma
\begin{lemma}
\label{lem:dual_cert}
Assume that $\|P_\Omega P_T \| \leq 1/2$. Then $(L_0,S_0)$ is the unique solution to PCP if $\exists (W, F)$ such that
\begin{equation}
\begin{aligned}
& UV^* + W = \lambda(\text{sign}(S_0) + F + P_\Omega D) \\
& P_T W = 0 \\
& \|W\| \leq 1/2 \\
& P_\Omega F = 0 \\
& \|F\|_\infty \leq 1/2 \\
&\|P_\Omega D \|_F \leq 1/4
\end{aligned}
\end{equation}

\end{lemma}

\begin{proof}
\end{proof}

By the previous Lemma, it suffices to produce a dual certificate $W$ such that
\begin{equation}
\begin{aligned}
& W \in T^{\perp} \\
& \|W\| < 1/2\\
& \|P_\Omega (UV^* - \lambda \sgn(S_0) + W) \|_F \leq \lambda / 4  \\
& \|P_{\Omega^\perp}(UV^* + W)\|_\infty < \lambda / 2
\end{aligned}
\end{equation}

since under these conditions, $D = \frac{1}{\lambda} P_\Omega (UV^* - \lambda \sgn(S_0) + W)$ and $F = \frac{1}{\lambda} P_{\Omega^\perp} (UV^* + W)$ satisfy the sufficient conditions given by Lemma~\ref{lem:dual_cert}. Indeed we have

\begin{align*}
UV^* + W - \lambda \sign(S_0) 
&= P_\Omega (UV^* + W - \lambda \sign(S_0))+ P_{\Omega^\perp} (UV^* + W - \lambda \sign(S_0)) \\
&= \lambda D+ P_{\Omega^\perp} (UV^* + W) \text{ since } \sign(S_0) \in \Omega\\
&= \lambda (D+F)
\end{align*}

and the first condition of Lemma~\ref{lem:dual_cert} is satisfied. The remaining conditions follow from the definition of $F$ and $D$.

%-----------------------------------------------------------------------------
\subsubsection{Bounding $\|P_\Omega P_T\|$}
 Under suitable conditions on the size of the support $\Omega_0$ of the sparse component, a bound can be derived on  $\|P_\Omega P_T\|$ \cite{Candes:2009uq}.
\begin{theorem}
Suppose $\Omega_0$ is sampled from the Bernoulli model with parameter $\rho_0$. Then with high probability,
\[
\|P_T - \rho_0^{-1} P_T P_{\Omega_0} P_T\| \leq \epsilon
\]
provided that $\rho_0 \geq C_0 \epsilon^{-2} \frac{\mu r \log n}{n}$ where $\mu$ is the incoherence parameter and $C_0$ is a numerical constant.
\end{theorem}

As a consequence, $\|P\Omega P_T\|$ can be bounded, and if $|\Omega|$ is not too large, then the desired bound $\|P_\Omega P_T\| \leq 1/2$ holds.



%-----------------------------------------------------------------------------------------------------------------------------------------------------------------
\subsection{Main proof}



%============================================================================================
\newpage
\section{Related Problems and Extensions}


%-----------------------------------------------------------------------------------------------------------------------------------------------------------------
\subsection{Exact Matrix completion}
Robust PCA is an extension of the exact matrix completion problem introduced in \cite{Candes:2009uq}, where one seeks to recover a low-rank matrix $L_0$ from a small fraction of its entries. More precisely, assume one is given $\{(L_0)_{ij}, (i,j)\in \Omega\}$ where $\Omega$ is a subset of $[n]\times [n]$.

Problem to solve

\begin{equation}
\begin{aligned}
&\text{minimize} && \text{rank}(L) \\
&\text{subject to} && P_\Omega L = P_\Omega L_0
\end{aligned}
\end{equation}


A heuristic is to minimize the nuclear norm of $L$

%-------------------------------------------------------------------------
\subsubsection{Incoherence}

Singular vectors have to be sufficiently spread

\begin{equation}
\mu(U) = \frac{n}{r} \max_i \|P_U e_i\|_2^2 = \frac{n}{r} \max_i \left[ \sum_{k=1}^r u_{ki}^2 \right]
\label{emc_incoherence}
\end{equation}

Assumptions:
\begin{itemize}
\item $\max \{\mu(U), \mu(V)\} \leq \mu_0$
\item $\left( \sum_k u_kv_k^*\right)_{ij} \leq \mu_1 \sqrt{\frac{r}{n_1 n_2}}$ (true for $\mu_1 = \mu_0\sqrt{r}$)
\item $m \geq c \max \{ \mu_1^*, \sqrt{\mu_0}\mu_1, \mu_0 n^{1/4}\}n r \beta \log n$
\end{itemize}

Under these assumptions, recovery is exact with high probability (at least $1-\frac{c}{n\beta}$)

Incoherent matrices:
\begin{itemize}
\item sampled from the incoherent basis model
\item sampled from the random orthogonal model: if $M = \sum_k \sigma_k u_k v_k^*$, then $\{u_1, \dots, u_r\}$ and $\{v_1, \dots, v_r\}$ are assumed to be selected at random.

\end{itemize}

%-------------------------------------------------------------------------
\subsubsection{Main result}




%-------------------------------------------------------------------------
\subsubsection{Comparing results to Robust PCA}
Robust PCA can be thought of as an extension of the matrix completion problem, where instead of having a known subset of the entries $\{(L_0)_{ij}, (i,j)\in \Omega\}$ and the rest is missing, we have an unknown subset of the entries and the rest is corrupted. In this sense, Robust PCA is a harder problem.

Note that the matrix $L_0$ can be recovered by Principal Component Pursuit, solving a different problem:
\begin{equation}
\begin{aligned}
&\text{minimize} && \|L\|_* + \lambda \|S\|_1\\
&\text{subject to} && P_\Omega (L+S) = P_\Omega L_0
\end{aligned}
\end{equation}


%TODO: numerical simulations



%-----------------------------------------------------------------------------------------------------------------------------------------------------------------
\subsection{Stable Principal Component Pursuit}
\subsubsection{Overview}

The paper studies the problem of recovering a low-rank matrix (the principal components) from a high- dimensional data matrix despite both small entry-wise noise and gross sparse errors. It proves that the solution to a convex program (a relaxation of classic Robust PCA) gives an estimate of the low-rank matrix that is simultaneously stable to small entry- wise noise and robust to gross sparse errors. The result shows that the proposed convex program recovers the low-rank matrix even though a positive fraction of its entries are arbitrarily corrupted, with an error bound proportional to the noise level. 

%-----------------------------------------------------------------------------
\subsubsection{Main result}

The paper consider a matrix $M\in\mathbb{R}^{n_1\times n_2}$ of the from $M = L_0+S_0+Z_0$, where~$L_0$ is (non-sparse) low rank, $S_0$ is sparse (modeling gross errors) and $Z_0$ is ``small'' (modeling a small noisy perturbation). The assumption on~$Z_0$ is simply that $||Z_0||_F \leq \delta$ for some small known~$\delta$. Hence at least for the theory part of the paper the authors do not assume anything about the distribution of the noise other than it is bounded (however they will gloss over this in their algorithm). 

The convex program to be solved is a slight modification of the standard Robust PCA problem and given by
\begin{align}
\begin{split}
\min_{L,S} \; &||L||_* + \lambda ||S||_1 \\
\text{s.t.} \quad &||M-L-S||_F \leq \delta
\end{split}
\label{mainresult:optproblem}
\end{align}
where $\lambda = 1/\sqrt{n_1}$. Under a standard incoherence assumption on~$L_0$ (which essentially means that $L_0$ should not be sparse) and a uniformity assumption on the sparsity pattern of~$S_0$ (which means that the support of~$S_0$ should not be too concentrated) the main result states that, with high probability in the support of~$S_0$, for any~$Z_0$ with $||Z_0||_F \leq \delta$, the solution $(\hat{L},\hat{S})$ to~\eqref{mainresult:optproblem} satisfies
\begin{align*}
||\hat{L}-L_0||_F^2 + ||\hat{S}-S_0||_F^2 \leq C n_1n_2\delta^2
\end{align*}
where~$C$ is a numerical constant. The above claim essentially states that the recovered low-rank matrix~$\hat{L}$ is stable with respect to non-sparse but small noise acting on all entries of the matrix.

In order to experimentally verify the predicted performance to their formulation, the author provide a comparison with an oracle. This oracle is assumed to provide information about the support of~$S_0$ and the row and column spaces of~$L_0$, which allows the computation of the MMSE estimator which otherwise would be computationally intractable (strictly speaking it of course is not really the MMSE, since it uses additional information from the oracle). Simulation results that show that the RMS error of the solution obtained through~\eqref{mainresult:optproblem} in the non-breakdown regime (that is, for the support of~$S_0$ sufficiently small) is only about twice as large as that of the oracle-based MMSE. This suggests that the proposed algorithm works quite well in practice. 



%-----------------------------------------------------------------------------
\subsubsection{Relations to existing work}

The result of the paper can be seen from two different view points. On the one hand, it can be interpreted from the point of view of standard PCA. In this case, the result states that standard PCA, which can in fact be shown to be statistically optimal w.r.t. i.i.d Gaussian perturbations, can also be made robust with respect to sparse gross corruptions. On the other hand, the result can be interpreted from the point of view of Robust PCA. In this case, it essentially states that the classic Robust PCA solution can itself be made robust with respect to some small but non-sparse noise acting on all entries of the matrix. 

Conceptually, the work presented in the paper is similar to the development of results for ``imperfect'' scenarios in compressive sensing where the measurements are noisy and the signal is not exact sparse. In this body of literature, $l_1$-norm minimization techniques are adapted to recover a vector $x_0 \in \mathbb{R}^n$ from contaminated observations $y=Ax_0+z$, where $A \in \mathbb{R}^{m\times n}$ with $m \ll n$ and z is the noise term. 


%-----------------------------------------------------------------------------
\subsubsection{Algorithm}

For the case of a noise matrix~$Z_0$ whose entries are i.i.d. $\Ncal(0,\sigma^2)$, the paper suggests to use an Accelerated Proximal Gradient (APG) algorithm (see algorithms section for details) for solving~\eqref{mainresult:optproblem}. Note that for~$\delta =0$ the problem reduces to the standard Robust PCA problem with an equality constraint on the matrices. For this case the APG algorithm proposed in~\cite{Lin:2009kx} solves an approximation of the form
\begin{align*}
\min_{L,S} \; &||L||_* + \lambda ||S||_1 + \frac{1}{2\mu} ||M-L-S||_F
\end{align*}
For the Stable PCP problem where~$\delta>0$ the authors advocate using the same algorithm with fixed but carefully chosen parameter~$\mu$ (similar to~\cite{Candes:2010fk}). In particular, they point out\footnote{this based on the strong Bai Yin Theorem~\cite{Bai:1988fk}, which implies that for an $n\times n$ real matrix with entries $\xi_{ij} \sim \Ncal(0,1)$ the it holds that $\limsup_{n\rightarrow \infty} \norm{Z_0}{2}{}/\sqrt{n} = 2$ almost surely}
 that for $Z_0 \in\Rbf^{n\times n}$ with~$(Z_0)_{ij} \sim \Ncal(0,\sigma^2)$ i.i.d. it holds that $n^{-1/2}\norm{Z_0}{2}{} \rightarrow \sqrt{2}\sigma$ almost surely as $n\rightarrow\infty$. They then choose the parameter~$\mu$ such that if $M=Z_0$, i.e. if $L_0=S_0=0$, the minimizer of the above problem is likely to be $\hat{L}=\hat{S}=0$. The claim is that this is the case for~$\mu = \sqrt{2n}\sigma$.
  
It is worth noting that the assumption of a Gaussian noise matrix~$Z_0$ is reasonable but not always satisfied. If it is not, then it is not clear if using the APG algorithm to solve the associated approximate problem is a good idea and different algorithms may be needed. The problem~\eqref{mainresult:optproblem} can be expressed as an SDP and can therefore in principle be solved using general purpose interior point solvers. However, the same scalability issues as in the standard Robust PCA problem will limit prohibit to use these methods for high-dimensional data. The paper~\cite{Aybat:2011vn} focuses on efficient first-order algorithms for solving~\eqref{mainresult:optproblem}.


%-----------------------------------------------------------------------------
\subsubsection{Conclusion and Outlook}

The paper addresses a problem of potentially very high practical relevance. While it is reasonable to assume that in many applications the low-rank component~$L_0$ will only be corrupted by a comparatively small number of gross errors (caused by rare and isolated events), the assumption of perfect measurements for the rest of the data outside the support of~$S_0$ that is made in classic Robust PCA will generally not hold for example due to sensor noise. This paper asserts that if the non-sparse noise component~$Z_0$ is sparse, then with high probability the recovered components are ``close'' to the actual ones. 

For simplicity, the paper models the non-sparse noise simply as an additive perturbation that is bounded in the Frobenius norm. In cases where one has additional information available about this noise, for example its distribution or some bounds on the absolute value of each entry, it might be possible to derive better bounds on the resulting errors. One possible extension could therefore be to look at exploiting structure in the noise. 

One thing the paper claims is that ``at a cost not so much higher than the classical PCA, [the] result is expected to have significant impact on many practical problems''. As mentioned above I do agree that the result has a significant impact on many practical problems. However, the claim concerning the computational complexity is very optimistic. The fastest solver for the special case $\delta =0$ (classic Robust PCA) currently seems to be a alternating directions augmented Lagrangian method. This method requires an SVD at each iteration, and for problems involving large-scale data the number of iterations can be very large. The standard PCP algorithm on the other hand is based on a single SVD, hence it can be computed much faster. 


%-----------------------------------------------------------------------------------------------------------------------------------------------------------------
\subsection{Robust Alignment by Sparse and Low-rank Decomposition}
\label{subsec: RASL}

The convex optimization framework for low-rank matrix recovery has been employed successfully. However, in practice, much more data can be viewed as low-rank only after some transformation is applied. The new formulation of this problem as Robust Alignment by Sparse and Low-rank Decomposition (RASL) \cite{Peng:2010}:
\begin{align}
\min_{A, E, \tau}  ||A||_{*} + \lambda||E||_{1} \quad  \text{s.t.} \;  D\circ\tau = A+E
\label{eq:rasl:original}
\end{align}
where $A  \in\mathbb{R}^{m\times n}$ is low-rank matrix, $A\in\mathbb{R}^{m\times n}$ is sparse matrix, $D$ is our measurements, which is the result of $(A+E)$ subjecting to transformation $\tau^{-1}$. Here we assume that the transformation is invertible. 
We define $D\circ\tau$ as: $D\circ\tau = [\;D_{1}\circ\tau_{1} \;|\;D_{2}\circ\tau_{2} \;|\; \dots \;|\; D_{n}\circ\tau_{n}\;]$, which is the measurements $D=[\;D_{1} \;|\;D_{2} \;|\; \dots \;|\; D_{n}\;] $subjects to set of transformations $\tau=[\;\tau_{1} \;|\;\tau_{2} \;|\; \dots \;|\; \tau_{n}\;] \in\mathbb{G}^n$, where $\mathbb{G}$ is a group of certain type of invertible transformations, which could be affine transform, rotation transform, etc.  \\

The main difficulty in solving ~\eqref{eq:rasl:original} is the nonlinearity of constraint $D\circ\tau = A+E$. When the change in $\tau$ is small, we can approximate this constraint by linearizing about the current estimate of $\tau$. Here, we assume that $\mathbb{G}$ is some $p$-parameter group and identify $\tau=[\;\tau_{1} \;|\;\tau_{2} \;|\; \dots \;|\; \tau_{n}\;] \in \mathbb{R}^{p\times n}$ with the parameterizations of all of the transformations. For $\Delta\tau = [\;\Delta\tau_{1} \;|\; \Delta\tau_{2} \;|\; \dots \;|\; \Delta\tau_{n}\;]$, write $D\circ(\tau+\Delta\tau) \approx D\circ\tau + \sum_{i=1}^n J_{i}\Delta\tau_{i}\epsilon_{i}$, where $J_{i} \doteq \frac{\partial}{\partial\zeta}(D_{i}\circ\zeta)|_{\zeta = \tau_{i}}$ is the Jacobian of the $i$-th measurement with respect to the transformation parameters $\tau_{i}$. $\{\epsilon_{i}\}$ denotes the standard basis for $\mathbb{R}^n$. This leads to a convex optimization problem in unknowns $A, E, \Delta\tau$:
\begin{align}
\min_{A, E, \Delta\tau}  ||A||_{*} + \lambda||E||_{1}  \quad \text{s.t.} \;  D\circ\tau + \sum_{i=1}^n J_{i}\Delta\tau\epsilon_{i}\epsilon_{i}^{T}= A+E
\label{eq:rasl:linearized}
\end{align}
It leads to algorithm~\ref{alg:RASL}




\begin{algorithm}[h!]
\label{alg:RASL}
\caption{RASL}
\KwIn{$D = [\; D_{1} \;|\; D_{2} \;|\; \dots \;|\; D_{n}]$, initial transformation $\tau_{1}, \tau_{2}, \dots, \tau_{n}$ in a certain parametric group $\mathbb{G}$, weight $\lambda > 0$.}
\While{not converged}{
\textbf{Step 1:} compute Jacobian matrices w.r.t. transformation: %$J_{i} \leftarrow \frac{\partial}{\partial\zeta}(D_{i}\circ\zeta)|_{\zeta = \tau_{i}}$ \\
\begin{equation}
J_{i} \leftarrow \frac{\partial}{\partial\zeta}(D_{i}\circ\zeta)|_{\zeta = \tau_{i}} \nonumber
\end{equation}
\textbf{Step 2 (inner loop):} solve the linearized convex optimization:
\begin{equation}
(A^{*}, E^{*}, \Delta\tau^{*}) \leftarrow \argmin_{A, E, \Delta\tau}  ||A||_{*} + \lambda||E||_{1}  \quad \text{s.t.} \;  D\circ\tau + \sum_{i=1}^n J_{i}\Delta\tau\epsilon_{i}\epsilon_{i}^{T}= A+E  \nonumber
\end{equation}
\textbf{Step 3:} update the transformation: $\tau \leftarrow \tau + \Delta\tau^{*}$
\newline
}
\KwOut{$A^{*}, E^{*}, \tau^{*}$}
\end{algorithm}

%============================================================================================
\section{Algorithms}


%============================================================================================
\section{Applications}


%============================================================================================
\newpage
\bibliographystyle{abbrv}
\bibliography{../common/RobustPCA}

\end{document}
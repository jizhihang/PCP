\documentclass{../common/projectreport}
\newcommand{\norm}[3]{{\left\|#1\right\|}_{#2}^{#3}}
\newcommand{\normball}[2]{{\mathbb{B}_{#1}^{#2}}}
\newcommand{\proximal}[2]{{\prox_{#1}(#2)}}
\newcommand{\forAll}{{\forall \,}}
\newcommand{\Rbb}{{\mathbb R}}
\newcommand{\Rbf}{{\mathbf R}}
\newcommand{\Lhat}{{\hat{L}}}
\newcommand{\Shat}{{\hat{S}}}
\newcommand{\Yhat}{{\hat{Y}}}
\newcommand{\Xhat}{{\hat{X}}}
\newcommand{\Nbb}{{\mathbb N}}
\newcommand{\Fbb}{{\mathbb F}}
\newcommand{\Cbb}{{\mathbb C}}
\newcommand{\Tbb}{{\mathbb T}}
\newcommand{\Zbb}{{\mathbb Z}}
\newcommand{\etabar}{{\bar{\eta}}}
\newcommand{\kbar}{{\bar{k}}}
\newcommand{\rbar}{{\bar{r}}}
\newcommand{\xibar}{{\bar{\xi}}}
\newcommand{\ubar}{{\bar{u}}}
\newcommand{\vbar}{{\bar{v}}}
\newcommand{\uhat}{{\hat{u}}}
\newcommand{\vbf}{{\mathbf{v}}}
\newcommand{\vhat}{{\hat{v}}}
\newcommand{\xbar}{{\bar{x}}}
\newcommand{\xhat}{{\hat{x}}}
\newcommand{\xdot}{{\dot{x}}}
\newcommand{\Xdot}{{\dot{X}}}
\newcommand{\xtilde}{{\tilde{x}}}
\newcommand{\xtildedot}{{\dot{\tilde{x}}}}
\newcommand{\Phitilde}{{\tilde{\Phi}}}
\newcommand{\phat}{{\hat{p}}}
\newcommand{\shat}{{\hat{s}}}
\newcommand{\xbf}{{\mathbf{x}}}
\newcommand{\wbf}{{\mathbf{w}}}
\newcommand{\Pbf}{{\mathbf{P}}}
\newcommand{\Xbf}{{\mathbf{X}}}
\newcommand{\Ebf}{{\mathbf{E}}}
\newcommand{\Obf}{{\mathbf{0}}}
\newcommand{\ebf}{{\mathbf{e}}}
\newcommand{\alphabf}{{\mathbf{\alpha}}} 
\newcommand{\alphabar}{{\bar{\alpha}}} 
\newcommand{\sgn}{\text{sgn}}
\newcommand{\Range}{\mathcal{R}}
\newcommand{\Nullspace}{\mathcal{N}}

\newcommand{\pderiv}[2]{\frac{\partial #1}{\partial #2}}
\newcommand{\pderivsq}[2]{\frac{\partial^2 #1}{\partial #2^2}}
\newcommand{\pderivxy}[3]{\frac{\partial^2 #1}{\partial #2 \partial #3}}

\newcommand{\One}{{\mathbf{1}}}

\newcommand{\Ccal}{{\mathcal{C}}}
\newcommand{\Dcal}{{\mathcal{D}}}
\newcommand{\Fcal}{{\mathcal{F}}}
\newcommand{\Hcal}{{\mathcal{H}}}
\newcommand{\Lcal}{{\mathcal{L}}}
\newcommand{\Ncal}{{\mathcal{N}}}
\newcommand{\Pcal}{{\mathcal{P}}}
\newcommand{\Qcal}{{\mathcal{Q}}}
\newcommand{\Rcal}{{\mathcal{R}}}
\newcommand{\Scal}{{\mathcal{S}}}
\newcommand{\Tcal}{{\mathcal{T}}}
\newcommand{\Ucal}{{\mathcal{U}}}
\newcommand{\Xcal}{{\mathcal{X}}}
\newcommand{\Ycal}{{\mathcal{Y}}}


\newcommand{\ifthen}[2]{\textbf{if} #1 \textbf{then} #2}
\newcommand{\ifthenreturn}[2]{\textbf{if} #1 \textbf{then return} #2}
\newcommand{\then}[1]{\textbf{then} #1}
\newcommand{\return}[1]{\textbf{return} #1}
\newcommand{\returns}[1]{\textbf{returns} #1}
\newcommand{\thenreturn}[1]{\textbf{then return} #1}
\newcommand{\foreachindo}[3]{\textbf{for each} #1 \textbf{in} #2 \textbf{do} #3}
\newcommand{\algindent}[1]{\STATE\hspace{#1\algorithmicindent}}

\DeclareMathOperator{\blkdiag}{blkdiag}
\DeclareMathOperator{\trace}{\mathbf{Tr}}
\DeclareMathOperator{\Convh}{Conv}
\DeclareMathOperator{\rank}{rk}
\DeclareMathOperator{\diag}{diag}
\DeclareMathOperator{\Realpart}{Re}
\DeclareMathOperator{\Imagpart}{Im}
\DeclareMathOperator{\Poles}{Poles}
\DeclareMathOperator{\spanspace}{span}
\DeclareMathOperator*{\argmax}{arg\,max}
\DeclareMathOperator*{\argmin}{arg\,min}
\DeclareMathOperator{\Onebf}{{\mathbf{1}}}
\DeclareMathOperator{\dom}{dom}
\DeclareMathOperator{\grad}{\nabla\!\!}
\DeclareMathOperator{\epi}{epi}
\DeclareMathOperator{\sign}{sign}
\DeclareMathOperator{\divergence}{div}
\DeclareMathOperator{\prox}{prox}

\newcommand\independent{\protect\mathpalette{\protect\independenT}{\perp}}
\def\independenT#1#2{\mathrel{\rlap{$#1#2$}\mkern2mu{#1#2}}}


\newcommand{\Acal}{{\mathcal{A}}}
\newcommand{\Abar}{{\bar{A}}}


\theoremstyle{plain}
\newtheorem{thm}{Theorem}
\theoremstyle{plain}
\newtheorem{fact}[thm]{Fact}
\theoremstyle{plain}
\newtheorem{prop}[thm]{Proposition}



%============================================================================================

\title{\Large Robust Principal Component Analysis}
\author{Maximilian Balandat \and Walid Krichene \and Chi Pang Lam \and Ka Kit Lam}

%============================================================================================
\begin{document}
\maketitle

%============================================================================================
\section{Introduction}
Given an observed matrix $M \in \Rbb^{n_1 \times n_2}$ that is formed as a superposition of a low-rank matrix $L_0$ and a sparse matrix $S_0$, 
\[
M = L_0 + S_0
\]
Robust Principal Component Analysis \cite{Candes:2011fk} is the problem of recovering the low-rank and sparse components. Under suitable assumptions on the rank and incoherence of $L_0$, and the distribution of the support of $S_0$, the components can be recovered exactly with high probability, by solving the Principal Component Pursuit (PCP) problem given by

\begin{equation}
\begin{aligned}
&\text{minimize} && \|L\|_* + \lambda \|S\|_1 \\
&\text{subject to} && L+S = M
\label{eq:pcp}
\end{aligned}
\end{equation}

Principal component pursuit minimizes a linear combination of the nuclear norm of a matrix $L$ and the $\ell_1$ norm of $M-L$. Minimizing the $\ell_1$ norm is known to favour sparsity, while minimizing the nuclear norm $\|L\|_* = \sum_{\sigma \in \sigma(L)} \sigma$ is known to favour low-rank matrices (intuitively, favours sparsity of the vector of singular values).

The low-rank component $S_0$ is viewed as a noise matrix, that can represent measurement noise, failure in some sensors that will result in completely corrupting a fraction of the observed entries, or missing data (which translates to having a fraction of the entries equal to zero). In this setting, one would like to be able to recover the original data $L_0$, without making assumptions on the magnitude $\|S_0\|_\infty$ of the sparse component. PCP achieves recovery with high probability in this setting, under alternate assumptions on the structure of $L_0$ and sparsity pattern of $S_0$.

One cannot expect to recover the components exactly in the most general case. Assume for example that $L_0$ is such that $(L_0)_{ij} = \delta_i^1\delta_j^1$, and $S_0 = -L_0$. Both matrices are sparse and low-rank, and clearly one cannot expect to recover the components in this case, since the observed matrix is $M = 0$. Therefore assumptions are made on the incoherence of $L_0$ and the support of $S_0$.

%-----------------------------------------------------------------------------------------------------------------------------------------------------------------
\subsection{Incoherence of the low rank component $L_0$}
The Incoherence conditions describe how much the singular vectors of a given matrix are aligned with the vectors of the canonical basis.

Let the SVD of $L_0$ be given by

\begin{align}
L_0 = U\Sigma V^* = \sum_{i=1}^r \sigma_i u_i v_i^*
\label{eq:svd}
\end{align}

where $U \in \Rbb^{n_1 \times r}$ and $V \in \Rbb^{n_2 \times r}$ are the matrices of left and right singular vectors respectivley, $U = [u_1, \dots, u_r]$, $V = [v_1, \dots, v_r]$. Then the incoherence conditions are given by

\begin{equation}
\begin{aligned}
\max_i \|U^*e_i\|_2^2 \leq \frac{\mu r}{n_1}, && \max_i \|V^*e_i\|_2^2 \leq \frac{\mu r}{n_2}
\label{eq:incoherence1}
\end{aligned}
\end{equation}

and

\begin{equation}
\| U V^* \|_\infty \leq \sqrt{\frac{\mu r}{n_1 n_2}}
\label{eq:incoherence2}
\end{equation}

Note that the condition $\|U^*e_i\|_2^2 \leq \frac{\mu r}{n_1}$ translates to $\sum_{k = 1}^r (u_k)_i^2 \leq \frac{\mu r}{n_1}$.  Also note that the orthogonal projection $P_U$ on $\text{Span}(u_1, \dots, u_r)$ is given by 
\[
UU^* = [u_1, \dots, u_r] 
\left[ \begin{array}{c} 
u_1^* \\ 
\vdots \\ 
u_r^* \\ 
\end{array} \right] 
= \sum_{k = 1}^r u_ku_k^*
\]
and the condition is equivalent to $\|P_U e_i \|_2^2 \leq \frac{\mu r}{n_1}$ since $\|U^* e_i \|_2^2 = e_i^*(UU^*)e_i = e_i^* P_U e_i = (e_i - P_Ue_i + P_U e_i )^*P_U e_i = \|P_U e_i\|_2^2$ ($P_U e_i$ and $e_i - P_U e_i$ are orthogonal). Or simply $\|P_U e_i\|_2^2 = e_i^* UU^*UU^*e_i = e_i^* UU^*e_i = \|U^*e_i\|_2^2$ since $U^*U = I_r$.

These conditions require the singular vectors to be ``spread'' enough with respect to the canonical basis. Intuitively, if the singular vectors of the low-rank matrix $L_0$ are aligned with a few canonical basis vectors, then $L_0$ will be sparse and hard to distinguish from the sparse corruption matrix $S_0$.



%-----------------------------------------------------------------------------------------------------------------------------------------------------------------
\subsection{Support of the sparse component $S_0$}
The cardinality of the support of $S_0$ is denoted $m$. Guaranteeing exact recovery requires $m$ to be small enough, in a sense that will be defined in the next section. Proving exact recovery will rely on a probabilistic argument on the distribution of sparse matrices $S_0$ on the set of matrices $\{ S \in \Rbb^{n_1 \times n_2} | \text{card}(\text{Supp}S) = m \}$assuming a uniform sampling model. The proof of the main result will use a different sampling model, and prove equivalence with the uniform model.


%============================================================================================

% !TEX root = report.tex

%============================================================================================
\section{Introduction}
Given an observed matrix $M \in \Rbb^{n_1 \times n_2}$ that is formed as a superposition of a low-rank matrix $L_0$ and a sparse matrix $S_0$,
\[
M = L_0 + S_0
\]
Robust Principal Component Analysis \cite{Candes:2011fk} is the problem of recovering the low-rank and sparse components. Under suitable assumptions on the rank and incoherence of $L_0$, and the distribution of the support of $S_0$, the components can be recovered exactly with high probability, by solving the Principal Component Pursuit (PCP) problem given by

\begin{equation}
\begin{aligned}
&\text{minimize} && \|L\|_* + \lambda \|S\|_1 \\
&\text{subject to} && L+S = M
\label{eq:pcp}
\end{aligned}
\end{equation}

Principal component pursuit minimizes a linear combination of the nuclear norm of a matrix $L$ and the $\ell_1$ norm of $M-L$. Minimizing the $\ell_1$ norm is known to favor sparsity, while minimizing the nuclear norm $\|L\|_* = \sum_{\sigma \in \sigma(L)} \sigma$ is known to favour low-rank matrices (intuitively, favors sparsity of the vector of singular values).

The low-rank component $S_0$ is viewed as a noise matrix, that can represent measurement noise, failure in some sensors that will result in completely corrupting a fraction of the observed entries, or missing data (which translates to having a fraction of the entries equal to zero). In this setting, one would like to be able to recover the original data $L_0$, without making assumptions on the magnitude $\|S_0\|_\infty$ of the sparse component, where $\|S\|_\infty = \max_{i,j} |S_{ij}|$. PCP achieves recovery with high probability in this setting, under alternate assumptions on the structure of $L_0$ and sparsity pattern of $S_0$.

One cannot expect to recover the components exactly in the most general case. Assume for example that $L_0$ is such that $(L_0)_{ij} = \delta_i^1\delta_j^1$, and $S_0 = -L_0$. Both matrices are sparse and low-rank, and clearly one cannot expect to recover the components in this case, since the observed matrix is $M = 0$. Therefore assumptions need to be made on the incoherence of $L_0$ and the support of $S_0$.


%-----------------------------------------------------------------------------------------------------------------------------------------------------------------
\subsection{Incoherence of the low rank component $L_0$}
The Incoherence conditions describe how much the singular vectors of a given matrix are aligned with the vectors of the canonical basis.

Let the SVD of $L_0$ be given by

\begin{align}
L_0 = U\Sigma V^* = \sum_{i=1}^r \sigma_i u_i v_i^*
\label{eq:svd}
\end{align}

where $U \in \Rbb^{n_1 \times r}$ and $V \in \Rbb^{n_2 \times r}$ are the matrices of left and right singular vectors respectivley, $U = [u_1, \dots, u_r]$, $V = [v_1, \dots, v_r]$. Then the incoherence conditions are given by

\begin{equation}
\begin{aligned}
\max_i \|U^*e_i\|_2^2 \leq \frac{\mu r}{n_1}, && \max_i \|V^*e_i\|_2^2 \leq \frac{\mu r}{n_2}
\label{eq:incoherence1}
\end{aligned}
\end{equation}

and

\begin{equation}
\| U V^* \|_\infty \leq \sqrt{\frac{\mu r}{n_1 n_2}}
\label{eq:incoherence2}
\end{equation}

Note that the condition $\|U^*e_i\|_2^2 \leq \frac{\mu r}{n_1}$ translates to $\sum_{k = 1}^r (u_k)_i^2 \leq \frac{\mu r}{n_1}$.  Also note that the orthogonal projection $P_U$ on $\text{Span}(u_1, \dots, u_r)$ is given by
\[
UU^* = [u_1, \dots, u_r]
\left[ \begin{array}{c}
u_1^* \\
\vdots \\
u_r^* \\
\end{array} \right]
= \sum_{k = 1}^r u_ku_k^*
\]
and the condition is equivalent to $\|P_U e_i \|_2^2 \leq \frac{\mu r}{n_1}$ since $\|U^* e_i \|_2^2 = e_i^*(UU^*)e_i = e_i^* P_U e_i = (e_i - P_Ue_i + P_U e_i )^*P_U e_i = \|P_U e_i\|_2^2$ ($P_U e_i$ and $e_i - P_U e_i$ are orthogonal). Or simply $\|P_U e_i\|_2^2 = e_i^* UU^*UU^*e_i = e_i^* UU^*e_i = \|U^*e_i\|_2^2$ since $U^*U = I_r$.

These conditions require the singular vectors to be ``spread'' enough with respect to the canonical basis. Intuitively, if the singular vectors of the low-rank matrix $L_0$ are aligned with a few canonical basis vectors, then $L_0$ will be sparse and hard to distinguish from the sparse corruption matrix $S_0$.



%-----------------------------------------------------------------------------------------------------------------------------------------------------------------
\subsection{Support of the sparse component $S_0$}
The cardinality of the support of $S_0$ is denoted $m$. Guaranteeing exact recovery requires $m$ to be small enough, in a sense that will be defined in the next section. Proving exact recovery will rely on a probabilistic argument which assumes the sparse matrix $S_0$ is drawn from a uniform distribution on the set of matrices with support of a fixed cardinality $m$, i.e. $\{ S \in \Rbb^{n_1 \times n_2} | \mathbf{card}(\mathbf{supp}(S)) = m \}$. Here $\mathbf{card}$ denotes the cardinality, and $\mathbf{supp}$ denotes the set of non zero elements, or support. The proof of the main result will use a different sampling model, and prove equivalence with the uniform model.


%============================================================================================
\section{Main Result}
\begin{theorem}
\label{thm:pcp}
Suppose $L_0 \in \mathbb{R}^{n \times n}$ satisfies incoherence conditions~(\ref{eq:incoherence1}) and (\ref{eq:incoherence2}) and that the support of $S_0$ is uniformly distributed among all sets of cardinality $m$. Then $\exists c$ such that with high probability over the choice of support of $S_0$ (at least $1-cn^{-10}$), Principal Component Pursuit with $m = 1/\sqrt{n}$ is exact, i.e. $\hat{L} = L_0$ and $hat{S} = S_0$ provided that

\begin{equation}
\begin{aligned}
\text{rank}(L_0) \leq \frac{\rho_r}{\mu} \frac{n}{(\log n)^2} && \text{ and } && m \leq \rho_s n^2
\end{aligned}
\end{equation}

\end{theorem}

Above, $\rho_r$ and $\rho_s$ are positive numerical constants. Note in particular that no assumptions are made on the magnitudes of the nonzero entries of $S_0$.

The first condition in the theorem bounds the rank of $L_0$, but also how spread the singular vectors have to be, since we need to have $\forall i$ (from the incoherence condition)
\[
\|U^*e_i\|_2^2 \leq \frac{\mu \text{rank}(L_0)}{n} \leq \frac{\rho_r}{(\log n)^2}
\]

The second condition bounds the size $m$ of the support of $S_0$.
%============================================================================================
\section{Proof}

The main arguments of the proof are the following:

First, change the model of the sparse matrix $S_0$ from the uniform sampling model, to the Bernoulli sampling model with fixed signs, then to the Bernoulli sampling model with random signs. To show equivalence of the results under the different sampling models, an elimination theorem is used.

Then using the random sign Bernoulli sampling model, it is shown that a dual certificate can be constructed with high probability, proving that $(L_0, S_0)$ is the unique optimizer, by constructing a subgradient that shows that any non-zero perturbation $H$ will result in a strict increase in the objective value $\|L_0 + H\|_* + \lambda \|S_0 - H\|_1$.

%-----------------------------------------------------------------------------------------------------------------------------------------------------------------
\subsection{Preliminaries}
\begin{itemize}
\item The subgradient of of the $\ell_1$ norm at $S_0$ supported on $\Omega$ is of the form $\sgn(S_0)+F$ where $P_\Omega F = 0$ and $\|F\|_\infty \leq 1$.
\item The subgradient of the nuclear norm (for details see Section~\ref{sub:nuclearnorm}) at $L_0 = U\Sigma V^*$ where $U, V \in \Rbb^{n \times r}$ and $\Sigma \in \Rbb^{r \times r}$, is of the form $U V^* + W$, where
\begin{equation}
\label{eq:nuc_norm_subg_cond_1}
\begin{aligned}
&U^*W = 0 \\
& WV = 0\\
& \|W\| \leq 1
\end{aligned}
\end{equation}

where $\|W\|$ denotes the operator norm of matrix $W$, i.e. $\|W\| = \underset{\|u\|_2 = 1}{\max} \|Wu\|_2 = \sigma_{\max}(W)$. Conditions~(\ref{eq:nuc_norm_subg_cond_1}) are equivalent to

\begin{equation}
\begin{aligned}
& P_T W = 0 \\
& \|W\| \leq 1
\end{aligned}
\end{equation}

where $T$ is the linear space of matrices defined by
\begin{equation}
\label{eq:T_subspace}
T = \{UX^* + YV^*, \  X, Y \in \mathbb{R}^{n \times r}\}
\end{equation}

Indeed, we have
\begin{align*}
P_TW = 0
&\Leftrightarrow W \in T^\perp  \\
&\Leftrightarrow \forall M \in T, \ Tr(W^*M) = 0\\
&\Leftrightarrow \forall X, Y \in \mathbb{R}^{n \times r}, \ Tr(W^*(UX^* + YV^*)) = 0\\
&\Leftrightarrow \forall X, Y \in \mathbb{R}^{n \times r}, \ Tr((U^*W)^*X^*) + Tr((WV)^*Y) = 0\\
&\Leftrightarrow U^*W = WV^* = 0
\end{align*}


\end{itemize}

Note that the projection on the orthogonal of $T$ is given by
\begin{equation}
P_{T^\perp} M = (I - UU^*)M(I - VV^*) \label{property: p1}
\end{equation}

\begin{proof}
Note that $UU^*$ is the orthogonal projection on the subspace spanned by the columns of $U$, and similarly for $VV^*$. Let $P_U = UU^*$, $P_{U^\perp} = I - UU^*$, and similarly for $V$.

Let $M_1 = (I - UU^*)M(I - VV^*) = P_{U^\perp} M P_{V^\perp}$. We have
\[
P_{T^\perp} M = M_1 \Leftrightarrow (M - M_1 \in T \text{ and } M_1 \perp M - M_1)
\]
and we have $M - M_1 = UU^*M + MVV^* - UU^*MVV^* = U(U^*M) + (MV - UU^*MV)V^* \in T$, and
\begin{align*}
Tr(M_1^* (M - M_1))
&= Tr((I-VV^*)M^*(I-UU^*)(UU^*M + MVV^* - UU^*MVV^*)) \\
&= Tr(P_{V^\perp}M^*P_{U^\perp}(P_UM + (M - UU^*M)P_V)) \\
&= Tr(P_{V^\perp}M^*P_{U^\perp}P_UM) + Tr((M - UU^*M)P_VP_{V^\perp}M^*P_{U^\perp})\\
&=0
\end{align*}
using the fact that $P_{U^\perp}P_U = P_V P_{V^\perp} = 0$ (projecting consecutively on a subspace and its orthogonal yields 0, or simply expanding, $(I-UU^*)UU^* = UU^* - UU^*UU^* = UU^* - U I_r U^* = 0$). This completes the proof.
\end{proof}

Note that since $P_{T^\perp}$ is an orthogonal projection, we have
\begin{equation}
\| P_{T^\perp} M \| \leq \|M\| \label{property: p2}
\end{equation}

and for any dyad $e_i e_j^*$, we have

%TODO: say why and where this is used?

\begin{align*}
\|P_{T^\perp} e_ie_j^*\|_F^2
&= Tr\left( (I-UU^*)e_ie_j^*(I-VV^*)(I-VV^*)^*e_je_i^*(I-UU^*)^* \right) \\
&= Tr\left( e_i^*(I-UU^*)^*(I-UU^*)e_ie_j^*(I-VV^*)(I-VV^*)^*e_j \right) \\
&= Tr\left( e_i^*(I-UU^*)^*(I-UU^*)e_i\right) Tr \left(e_j^*(I-VV^*)(I-VV^*)^*e_j \right) \\
&= \|(I - UU^*)e_i\|_2^2 \|(I - VV^*)e_j\|_2^2
\end{align*}

and since $UU^*$ is an orthogonal projection, we have
\begin{align*}
\|(I - UU^*)e_i\|_2^2
&= \|e_i\|_2^2 - \|UU^*e_i\|_2^2 \\
&\geq 1 - \mu r / n
\end{align*}

where the last inequality results form the incoherence condition~(\ref{eq:incoherence2}), $\|U^*e_i\|_2^2 \leq \frac{\mu r}{n}$. Therefore

\begin{equation}
\|P_{T^\perp} e_ie_j^*\|_F^2 \geq (1 - \mu r / n)^2   \label{property: p3}
\end{equation}

equivalently, using the fact that $\|P_{T^\perp} e_ie_j^*\|_F^2 + \|P_{T} e_ie_j^*\|_F^2 = \|e_ie_j^*\|_F^2 = 1$, we have

\begin{align*}
\|P_{T} e_ie_j^*\|_F^2
&\leq 1 - \left(1 - \frac{\mu r}{n} \right)^2 \\
&= \frac{2\mu r}{n} - \left( \frac{\mu r}{n} \right)^2 \\
&\leq \frac{2\mu r}{n}
\end{align*}

Thus
\begin{equation}
  \|P_{T} e_ie_j^*\|_F^2 \leq \frac{2\mu r}{n}   \label{property: p4}
\end{equation}
%-----------------------------------------------------------------------------------------------------------------------------------------------------------------
\subsection{Elimination Theorem}
The following elimination theorem states the intuitive fact that if PCP exactly recovers the components of $M = L+S$, then it also exactly recovers the components of $M = L+S'$ where $S'$ is a trimmed version of $S$ ($\text{supp}(S')\subset \text{supp}(S)$ and $S$ and $S'$ coincide on $\text{supp}(S')$)

\begin{theorem}
Suppose the solution to the PCP problem (\ref{eq:pcp}) with input data $M_0 = L_0 + S_0$ is unique and exact, and consider $M_0' = L_0 + S_0'$ where $S_0'$ is a trimmed version of $S_0$. Then the solution to (\ref{eq:pcp}) with input $M_0'$ is exact as well.
\end{theorem}

\begin{proof}
Let $S_0' = P_{\Omega_0} S_0$ and let $(\hat{L}, \hat{S})$ be the solution to (\ref{eq:pcp}) with input $L_0 + S_0'$. Then since $(L_0, S_0')$ is a feasible point for (\ref{eq:pcp}), it provides un upper bound on the optimal value
\[
\|\hat{L}\|_* + \lambda \|\hat{S}\|_1 \leq \|L_0\|_* + \lambda \|S_0'\|_1
\]
then decomposing $S_0$ into the orthogonal components $S_0 = P_{\Omega_0} S_0 + P_{\Omega_0^\perp} S_0 = S_0' + P_{\Omega_0^\perp} S_0$, we have $\|S_0'\|_1 = \|S_0\|_1 - \|P_{\Omega_0^\perp} S_0\|_1$, thus we have

\[
\|\hat{L}\|_* + \lambda \|\hat{S}\|_1 + \lambda \|P_{\Omega_0^\perp} S_0\|_1 \leq \|L_0\|_* + \lambda \|S_0\|_1
\]

and using the triangular inequality

\[
\|\hat{L}\|_* + \lambda \|\hat{S} + P_{\Omega_0^\perp} S_0\|_1 \leq \|L_0\|_* + \lambda \|S_0\|_1
\]

we observe that $(\hat{L}, \hat{S} + P_{\Omega_0^\perp} S_0)$ is feasible for the problem with input $M = L_0 + S_0$, for which the optimal value is precisely $\|L_0\|_* + \lambda \|S_0\|_1$. Therefore by uniqueness of the solution, we have
\[
\begin{aligned}
&\hat{L} = L_0 \\
&\hat{S} + P_{\Omega_0^\perp} S_0 = S_0
\end{aligned}
\]

the second equality is equivalent to $\hat{S} = S_0 - P_{\Omega_0^\perp} S_0 = P_{\Omega_0} S_0 = S_0'$. This completes the proof.
\end{proof}


%-----------------------------------------------------------------------------------------------------------------------------------------------------------------
\subsection{Derandomization}
Derandomization is used to show equivalence between the problem where the signs of the entries of $S_0$ are random, and the problem where the entries of $S_0$ have fixed signs.

In the setting of Theorem~\ref{thm:pcp}, the non-zero entries of the sparse component $S_0$ are fixed, but the proof will use a stronger assumption: the signs of the non-zero entries are independent Bernoulli variables. The following theorem shows equivalence of the two settings.

\begin{theorem}
Suppose $L_0$ satisfies conditions of Theorem~\ref{thm:pcp}, and that the support of $S_0$ is sampled from a Bernoulli model with parameter $2\rho_s$, and the signs of $S_0$ are i.i.d. Bernoulli $\pm 1$ with parameter $\frac{1}{2}$, and independent from the support. Then:

If the PCP solution is exact with high probability, then it is exact with at least the same probability for the model in which signs of $S_0$ are fixed and the support is sampled from a Bernoulli distribution with parameter $\rho_s$.
\end{theorem}

\begin{proof}
Consider the fixed values model, and let $S_0 = P_\Omega S$ for some matrix $S$, and the support $\Omega$ is sampled from a Bernoulli distribution. Thus the components of $S_0$ are independent and
\[
(S_0)_{ij} =
\begin{cases}
S_{ij} & \text{w.p. } \rho_s \\
0 & \text{w.p. } 1 - \rho_s
\end{cases}
\]

the idea of the proof is to craft a new model, and show that it is equivalent (in terms of probability distribution) to the above model.

Let $E$ be a random sign matrix, with i.i.d. entries
\[
E_{ij} =
\begin{cases}
1 & \text{w.p. } \rho_s \\
0 & \text{w.p. } 1 - 2\rho_s\\
-1 & \text{w.p. } \rho_s
\end{cases}
\]
and $\Delta(E)$ an elimination matrix, function of $E$, defined as
\[
\Delta_{ij} =
\begin{cases}
0 & \text{if } E_{ij}S_{ij} < 0 \ (E_{ij} \text{ and } S_{ij} \text{ have different signs} )\\
1 & \text{otherwise}
\end{cases}
\]

the entries of $\Delta$ are functions of independent variables, and are therefore independent.

Now consider the following variable
\[
S'_0 = \Delta \circ |S| \circ E
\]
where $\circ$ is the component wise product. Then $S_0$ and $S'_0$ have the same distribution. Indeed, it suffices by independence to check that they have the same marginals:

\begin{align*}
P((S'_0)_{ij} = S_{ij})
&= P(\Delta_{ij} = 1 \text{ and } E_{ij} = \sgn(S_{ij}) ) \\
&= P(E_{ij}S_{ij} \geq 0 \text{ and } E_{ij} = \sgn(S_{ij})) \\
&= P(E_{ij} = \sgn S_{ij}) \\
&= \rho_s
\end{align*}
and
\[
P(S_0 = S_{ij}) = \rho_s
\]

Finally, since, by assumption, PCP recovers $|S| \circ E$ with high probability, then by the elimination theorem, it also recovers $\Delta \circ |S| \circ E$ with at least the same probability. The result follows since $S'_0$ and $S_0$ have the same distribution.
\end{proof}

We remark that the uniform sampling and the iid Bernoulli sampling model are indeed equivalent and the justification is given in Section~\ref{sub:bernoullisampling}.
%-----------------------------------------------------------------------------------------------------------------------------------------------------------------
\subsection{Dual certificate}
The following lemma gives a simple sufficient condition for the pair $(L_0,S_0)$ to be the unique optimal solution to PCP.

\begin{lemma}
Assume that $\|P_\Omega P_T \| < 1$. Then $(L_0,S_0)$ is the unique solution to PCP if $\exists (W, F)$ such that
\begin{equation}
\begin{aligned}
& UV^* + W = \lambda(\sgn(S_0) + F) \\
& P_T W = 0 \\
& \|W\|<1 \\
& P_\Omega F = 0 \\
& \|F\|_\infty < 1
\end{aligned}
\end{equation}

\end{lemma}


\begin{proof}
We first prove that the condition $\|P_\Omega P_T\| < 1$ is equivalent to $\Omega \cap T = \{0\}$.

First, if $\Omega \cap T \neq \{0\}$, then let $M_0 \in \Omega \cap T$, $M_0 \neq 0$. We have $\| P_\Omega P_T M_0\| = \|M_0\|$, thus $\| P_\Omega P_T\| = \underset{M \neq 0}{\max} \frac{\| P_\Omega P_T M\|}{\|M\|} \geq 1$.

Conversely, if $\| P_\Omega P_T\| \geq 1$, then $\exists M_0 \neq 0$ such that $\|M_0\| \leq \| P_\Omega P_T M_0\|$. But since $P_\Omega$ and $P_T$ are orthogonal projections, we have $\|M_0\| \leq \| P_\Omega P_T\| \leq \|P_T M_0\| \leq \|M_0\|$, where inequalities must hold with equality. In particular, we have $\|P_T M_0\| = \|M_0\|$, which implies $P_T M_0 = M_0$ (to prove this, decompose $\|M_0\|$ into the orthogonal components $\|M_0\|^2 = \|M_0 - P_TM_0\|^2 + \|P_TM_0\|^2$, thus $\|P_TM_0\| = \|M_0\| \Rightarrow \|M_0 - P_TM_0\| = 0 \Rightarrow M_0 = P_T M_0$), then similarly, $\|P_\Omega M_0\| = \|M_0\|$, which implies $P_\Omega M_0 = M_0$. Therefore $M_0 \in \Omega \cap T$. This proves the equivalence $\|P_\Omega P_T\| < 1 \Leftrightarrow \Omega \cap T = \{0\}$.



To prove that $(L_0, S_0)$ is the unique optimizer, we show that for any feasible perturbation $(L_0 + H, S_0 - H)$ where $H \neq 0$ strictly increases the objective. Let
\begin{itemize}
\item $UV^* + W_0$ be an arbitrary subgradient of the nuclear norm at $L_0$, where $\|W_0\| \leq 1$ and $P_T W_0 = 0$
\item $\sgn (S_0) + F_0$ be an arbitrary subgradient of the $\ell_1$-norm at $S_0$, where $\|F_0\|_\infty \leq 1$ and $P_\Omega F_0 = 0$
\end{itemize}

Then we can lower bound the value of the objective
\[
\|L_0 + H\|_* + \lambda \|S_0 - H\|_1 \geq \|L_0\|_* + \lambda \|S_0\|_1 + \langle UV^* + W_0, H\rangle - \lambda \langle \sgn(S_0) + F_0, H\rangle
\]

Now we pick a particular pair $(W_0, F_0)$ such that
\begin{itemize}
\item $\langle W_0, H \rangle = \|P_{T^\perp} H\|_*$, for example $W_0 = P_{T^\perp} W$ where $W$ is a normed matrix such that $\langle W, P_{T^\perp} H \rangle = \|P_{T^\perp} H\|_*$ (by duality of $\|.\|$ and $\|.\|_*$)
\item $\langle F_0, H \rangle = -\|P_{\Omega^\perp} H\|_1$, for example $F_0 = -\sgn(P_{\Omega^\perp} H)$
\end{itemize}
then we have

\[
\|L_0 + H\|_* + \lambda \|S_0 - H\|_1 \geq \|L_0\|_* + \lambda \|S_0\|_1  + \|P_{T^\perp} H\|_* + \|P_{\Omega^\perp} H\|_1 + \langle UV^* - \lambda  \sgn(S_0), H\rangle
\]

we can bound the inner product using the definition of $W$ and $F$,

\begin{align*}
|\langle UV^* - \lambda  \sgn(S_0), H\rangle|
& = |\langle \lambda F - W, H\rangle|  & \text{since } UV^* + W = \lambda(\sgn(S_0) + F) \\
&\leq |\langle W, H \rangle| + \lambda |\langle F, H \rangle| & \text{by the triangular inequality} \\
&\leq \beta (\|P_{T^\perp} H\|_* + \lambda \|P_{\Omega^\perp} H\|_1)
\end{align*}
where $\beta = \max (\|W\|, \|F\|_\infty) < 1$, and the last inequality follows from the fact that
\[
\begin{aligned}
& \|P_{T^\perp} H\|_* \geq \langle P_{T^\perp} H, W/\|W\| \rangle \geq \langle H, W/\|W\| \rangle\\
& \|P_{\Omega^\perp} H\|_1 \geq \langle P_{\Omega^\perp} H, F/\|F\|_\infty \rangle \geq \langle H, F/\|F\|_\infty \rangle
\end{aligned}
\]
Thus
\begin{align*}
\|L_0 + H\|_* + \lambda \|S_0 - H\|_1 - \|L_0\|_* - \lambda \|S_0 \|_1
&\geq  (1-\beta) \left( \|P_{T^\perp} H\|_* + \lambda \|P_{\Omega^\perp} H\|_1 \right) \\
& > 0
\end{align*}
since $\|P_{T^\perp} H\|_* = \|P_{\Omega^\perp} H\|_1 = 0$ only if $P_{T^\perp} H = P_{\Omega^\perp} H = 0$, i.e. $H \in \Omega \cap T$, and, by assumption, $\Omega \cap T = {0}$ and $H \neq 0$. Therefore the objective strictly increases with a non-zero perturbation. This completes the proof.

\end{proof}


The proof of the main theorem will use a slightly different result, given by the following Lemma
\begin{lemma}
\label{lem:dual_cert}
Assume that $\|P_\Omega P_T \| \leq 1/2$. Then $(L_0,S_0)$ is the unique solution to PCP if $\exists (W, F)$ such that

\begin{equation}
\label{eq:dual_cert_cond}
\begin{aligned}
& UV^* + W = \lambda(\text{sign}(S_0) + F + P_\Omega D) \\
& P_T W = 0 \\
& \|W\| \leq 1/2 \\
& P_\Omega F = 0 \\
& \|F\|_\infty \leq 1/2 \\
&\|P_\Omega D \|_F \leq 1/4
\end{aligned}
\end{equation}

\end{lemma}

\begin{proof}
Using $\beta = \max (\|W\|, \|F\|_\infty) \leq \frac{1}{2}$ in the previous proof, we have for a non-zero perturbation $H$
\begin{align*}
\|L_0 + H\|_* + \lambda \|S_0 - H\|_1 - \|L_0\|_* - \lambda \|S_0 \|_1
&\geq  \frac{1}{2} \left( \|P_{T^\perp} H\|_* + \lambda \|P_{\Omega^\perp} H\|_1 \right) - \lambda \langle P_{\Omega} D, H\rangle \\
& \geq \frac{1}{2} \left( \|P_{T^\perp} H\|_* + \lambda \|P_{\Omega^\perp} H\|_1 \right) - \frac{\lambda}{4} \|P_{\Omega} H\|_F
\end{align*}

the last term can be further bounded
\begin{align*}
\|P_\Omega H\|_F
&\leq \|P_\Omega P_T H\|_F + \|P_\Omega P_{T^\perp} H\|_F \\
&\leq \frac{1}{2} \|H\|_F + \|P_{T^\perp} H\|_F & \text{using } \|P_\Omega P_T\| \leq \frac{1}{2} \text{ and } \|P_\Omega\| \leq 1 \\
&\leq \frac{1}{2} \|P_\Omega H\|_F + \frac{1}{2} \|P_{\Omega^\perp} H\|_F + \|P_{T^\perp} H\|_F
\end{align*}

therefore

\[
\|P_\Omega H\|_F \leq \|P_{\Omega^\perp} H\|_F + 2 \|P_{T^\perp} H\|_F
\]

and we conclude by lower bounding the increase in the objective
\begin{align*}
\|L_0 + H\|_* + \lambda \|S_0 - H\|_1 - \|L_0\|_* - \lambda \|S_0 \|_1
&\geq  \frac{1}{2} \left( (1- \lambda) \|P_{T^\perp} H\|_* + \frac{\lambda}{2} \|P_{\Omega^\perp} H\|_1 \right) \\
& > 0
\end{align*}
since $ \|P_{T^\perp} H\|_* =  \|P_{\Omega^\perp} H\|_1 = 0$ only if $P_{\Omega^\perp} H = P_{T^\perp} H = 0$, i.e. $H \in \Omega \cap T$, and, by assumption, $\Omega \cap T = \{ 0 \}$ ($\|P_\Omega P_T\| \leq \frac{1}{2} < 1$).
This completes the proof.

\end{proof}

\iffalse
By the previous Lemma, it suffices to produce a dual certificate $W$ such that
\begin{equation}
\begin{aligned}
& W \in T^{\perp} \\
& \|W\| < 1/2\\
& \|P_\Omega (UV^* - \lambda \sgn(S_0) + W) \|_F \leq \lambda / 4  \\
& \|P_{\Omega^\perp}(UV^* + W)\|_\infty < \lambda / 2
\end{aligned}
\end{equation}

since under these conditions, $D = \frac{1}{\lambda} P_\Omega (UV^* - \lambda \sgn(S_0) + W)$ and $F = \frac{1}{\lambda} P_{\Omega^\perp} (UV^* + W)$ satisfy the sufficient conditions given by Lemma~\ref{lem:dual_cert}. Indeed we have

\begin{align*}
UV^* + W - \lambda \sign(S_0)
&= P_\Omega (UV^* + W - \lambda \sign(S_0))+ P_{\Omega^\perp} (UV^* + W - \lambda \sign(S_0)) \\
&= \lambda D+ P_{\Omega^\perp} (UV^* + W) \text{ since } \sign(S_0) \in \Omega\\
&= \lambda (D+F)
\end{align*}

and the first condition of Lemma~\ref{lem:dual_cert} is satisfied. The remaining conditions follow from the definition of $F$ and $D$.
\fi

%-----------------------------------------------------------------------------
\subsubsection{Bounding $\|P_\Omega P_T\|$}
 Under suitable conditions on the size of the support $\Omega_0$ of the sparse component, a bound can be derived on  $\|P_\Omega P_T\|$ \cite{Candes:2009uq}.
\begin{theorem}
Suppose $\Omega_0$ is sampled from the Bernoulli model with parameter $\rho_0$. Then with high probability,
\[
\|P_T - \rho_0^{-1} P_T P_{\Omega_0} P_T\| \leq \epsilon
\]
provided that $\rho_0 \geq C_0 \epsilon^{-2} \frac{\mu r \log n}{n}$ where $\mu$ is the incoherence parameter and $C_0$ is a numerical constant.
\end{theorem}

As a consequence, $\|P_\Omega P_T\|$ can be bounded, and if $|\Omega|$ is not too large, then the desired bound $\|P_\Omega P_T\| \leq 1/2$ holds.

\subsection{Probabilistic Guarantee via Dual Certification}

We now present the proof that, under the assumptions of Robust PCA, then with high probability, we can find a dual certificate $(W,F,D)$ that satisfies the conditions~(\ref{eq:dual_cert_cond}) of Lemma~\ref{lem:dual_cert}. This will achieve the proof of the probabilistic guarantee of recovery of $(L_0, S_0)$.

In order to construct a dual certificate $(W, F, D)$, we define $W = W^{L}+W^{S}$ where $W^{L}$ and $W^{S}$ are constructed to satisfy the following
properties. First, $P_{\Omega}(W^{S}) = \lambda \sign(S_0)$. Then it also need to satisfies the following.

\begin{lemma} \label{wl} Let $S_0\sim Bern(\rho)$ iid for each entry with $\Omega$
as its support set. Set $j_{0}=2\log n$. With the assumptions in
the main theorem of RPCA, $W^{L}$ satisfies the following with high
probability.
\begin{enumerate}
\item $||W^{L}||<\frac{1}{4}$
\item $||P_{\Omega}(UV^{*}+W^{L})||_{F}<\frac{\lambda}{4}$
\item $||P_{\Omega^{\bot}}(UV^{*}+W^{L})||_{\infty}<\frac{\lambda}{4}$
\end{enumerate}
\end{lemma}

\begin{lemma} \label{ws} Let $S_0\sim Bern(\rho)$ iid for each entry with $\Omega$
as its support set. With the assumptions in the main theorem of RPCA,
$W^{S}$ satisfies the following with high probability.
\begin{enumerate}
\item $||W^{S}||<\frac{1}{4}$
\item $||P_{\Omega^{\bot}}(W^{S})||_{\infty}<\frac{\lambda}{4}$
\end{enumerate}
\end{lemma}

After having these two lemmas, we are ready to show that they help to justify the probabilistic guarantee. We note that
\begin{eqnarray*}
UV^{*}+W & = & W^{L}+W^{S}+UV^{*}\\
 & = & \lambda \left[ P_{\Omega} \left( \frac{UV^{*}+W^{L}}{\lambda} \right) + sign(S_{0}) + P_{\Omega^{\bot}} \left( \frac{W^{L}+W^{S}+UV^{*}}{\lambda} \right) \right]
\end{eqnarray*}
Thus we take
\begin{eqnarray*}
D & = & \frac{UV^{*}+W^{L}}{\lambda}\\
W & = & W^{L}+W^{S}\\
F & = & P_{\Omega} \left( \frac{W^{L}+W^{S}+UV^{*}}{\lambda} \right)
\end{eqnarray*}


From Lemma~(\ref{wl}) and Lemma~(\ref{ws}), we can check that $(W,F,D)$ satisfies the conditions of Lemma~(\ref{lem:dual_cert}), thus establishing the probabilistic guarantee.

The following section will discuss in depth how to construct $W^{L}$ and $W^{S}$ that satisfy conditions of Lemma~(\ref{wl}) and Lemma~(\ref{ws}). 

%----------------------------------------------------------------------------------------------------------------------------------------------------------------

% !TEX root = report.tex
\newpage
%-----------------------------------------------------------------------------------------------------------------------------------------------------------------
\subsection{Proof of the Lemma about golfing scheme and dual certificate }
Golfing scheme:

The golfing scheme involves creating a $W^{L}$ according to the following method.
\begin{enumerate}
\item Fix $j_{0}\ge1$, define $\Omega_{j}\sim Bern(q)$ iid with $1\le j\le j_{0}$ and $\rho=(1-q)^{j_{0}}$. Define the complement of support of $\Omega$ by $\Omega=\cup_{1\le j\le j_{0}}\Omega_{j}^{C}$.
\item Define a sequence of matrix which finally ends at $W^{L}$
\begin{enumerate}
\item $Y_{0}=0$
\item $Y_{j}=Y_{j-1}+\frac{1}{q}P_{\Omega_{j}}P_{T}(UV^{*}-Y_{j-1})$ for $1\le j\le j_{0}$
\item $W^{L}=P_{\ensuremath{T^{\bot}}}(Y_{j_{0}})$
\end{enumerate}
\end{enumerate}

We first list a number of facts that will be used in the proof of Lemma~(\ref{wl}).\\

\begin{fact}[\cite{Candes:2011fk}]
\label{fact2}
If we fix $Z\in T$, $\Omega_{0}\sim Bern(\rho_{0})$, and $\rho_{0}\ge C_{0}\epsilon^{-2}\frac{\mu r\log n}{n}$ , then with high probability, we will have,

\label{fact5}
\[
\|Z-\rho_{0}^{-1}P_{T}P_{\Omega_{0}}(Z)\|_{\infty}\le\epsilon\|Z\|_{\infty}
\]

\begin{fact}[\cite{Candes:2011fk}]
\label{fact3}
If we fix $Z$, $\Omega_{0}\sim Bern(\rho_{0})$, and $\rho_{0}\ge C_{0}\frac{\mu\log n}{n}$, then with high probability,
we will have,
\begin{eqnarray*}
\|(I-\rho_{0}^{-1}P_{\Omega_{0}})Z\| & \le & C_{0}^{'}\sqrt{\frac{n\log n}{\rho_{0}}}\|Z\|_{\infty}
\end{eqnarray*}

\begin{fact}[\cite{Candes:2011fk}]
\label{fact4}
If $\Omega_{0}\sim Bern(\rho_{0})$,$\rho_{0}\ge C_{0}\epsilon^{-2}\frac{\mu r\log n}{n}$, then with high probability, we will have,

\[
\|P_{T}-\rho_{0}^{-1}P_{T}P_{\Omega_{0}}P_{T}\|\le\epsilon
\]

\begin{fact}[\cite{Candes:2011fk}]
If $\Omega \sim Bern(\rho)$ and $1-\rho\ge C_{0}\epsilon^{-2}\frac{\mu r\log n}{n}$, then with high probability $\|P_{\Omega}P_{T}\|^{2}\le\rho+\epsilon$
\end{fact}
\end{fact}
\end{fact}
\end{fact}


Now we present the proof of Lemma (\ref{wl})

\begin{proof}
We define another sequence of matrix $Z_{j}=UV^{*}-P_{T}(Y_{j})$. There are some properties about~$Z_{j}$ which allows us to establish the proof. We survey them here and provides the proof of them.

i) Note that
\begin{equation}
Z_{j} = \left( P_{T}-\frac{1}{q}P_{T}P_{\Omega_{j}}P_{T} \right) Z_{j-1}. \label{wlp1}
\end{equation}
The reason is as follows.

\[
\begin{aligned}
Z_{j}
 & = UV^{*} - P_{T} \left( Y_{j-1} + \frac{1}{q}P_{\Omega_{j}} P_{T}(UV^{*}-Y_{j-1}) \right)
 &&\text{by construction of } Y_j \\
 & = UV^{*} - P_{T} ( Y_{j-1}) - \frac{1}{q}P_{T} P_{\Omega_{j}} P_{T}( UV^{*} - Y_{j-1} )
 &&\text{by of linearity of } P_{T}\\
 & = Z_{j-1} - q^{-1}(P_{T}P_{\Omega_{j}}(UV^{*}-P_{T}(Y_{j-1})))
 &&\text{since } P_{T}(UV^{*}) = UV^{*}\\
 & = P_{T}(Z_{j-1}) - q^{-1}(P_{T}P_{\Omega_{j}}P_{T}Z_{j-1})
 &&\text{since } Z_{j-1}\in T\\
 & = ( P_{T} - q^{-1} P_{T} P_{\Omega_{j}} P_{T} ) Z_{j-1}
 &&\text{by linearity}
\end{aligned}
\]

ii) If $q\ge C_{0}\epsilon^{-2}\frac{\mu r\log n}{n}$, then  with high probability,
%
\begin{equation}
\| Z_{j}\|_{\infty}\le\epsilon^{j}\|UV^{*}\|_{\infty} \label{wlp2}
\end{equation}

The reason is as follows. By Fact (\ref{fact2}), we have,
%
\begin{eqnarray*}
\|Z_{j-1}-q^{-1}P_{T}P_{\Omega_{j}}Z_{j-1}\|_{\infty} & \le & \epsilon\|Z_{j-1}\|_{\infty} \\
\|Z_{j}\|_{\infty} & \le & \epsilon\|Z_{j-1}\|_{\infty} \text{by (\ref{wlp1})}
\end{eqnarray*}


Inductively, we get the desired.
%
iii) If $q\ge C_{0}\epsilon^{-2}\frac{\mu r\log n}{n}$, then
\begin{equation}
\|Z_{j}\|_{F}\le\epsilon^{j}\sqrt{r}  \label{wlp3}
\end{equation}

The reason is as follows. By Fact~(\ref{fact4}), we have,
%
\begin{align*}
\left\|(P_{T}-q^{-1}P_{T}P_{\Omega_{0}}P_{T}) \left( \frac{Z_{j-1}}{\|Z_{j-1}\|_{F}} \right) \right\|_{F} & \le \epsilon\\
\|(P_{T}-q^{-1}P_{T}P_{\Omega_{0}}P_{T})Z_{j-1}\|_{F} & \le \epsilon\|Z_{j-1}\|_{F} && \text{by rearranging terms}\\
\|Z_{j}\|_{F} & \le \epsilon\|Z_{j-1}\|_{F} &&\text{by (\ref{wlp1}) }
\end{align*}

Inductively, we get the desired result.

After establishing these properties, we are ready to prove that golfing scheme yields $W^{L}$ that satisfies the desired properties.

1) Proof of condition (1):
\[
\begin{aligned}
\|W^{L}\| & = \|P_{T^{\bot}}(Y_{j_{0}})\| && \text{by definition}\\
 & \le \sum_{j=1}^{j_{0}}\frac{1}{q}\|P_{T^{\bot}}P_{\Omega_{j}}Z_{j-1}\|
 &&\text{since } Y_{j} = Y_{j-1} + q^{-1} P_{\Omega_{j}}(Z_{j-1})\\
 & = \sum_{j=1}^{j_{0}}\|P_{T^{\bot}}(\frac{1}{q}P_{\Omega_{j}}Z_{j-1}-Z_{j-1})\|
 &&\text{since }Z_{j}\in T\\
 & \le \sum_{j=1}^{j_{0}}\|(\frac{1}{q}P_{\Omega_{j}}Z_{j-1}-Z_{j-1})\|
 &&\text{since }\|P_{T^{\bot}}(M)\|\le\|M\|\\
 & \le C_{0}^{'}\sqrt{\frac{n\log n}{q}}\sum_{j=1}^{j_{0}}\|Z_{j-1}\|_{\infty}
 &&\text{by Fact(\ref{fact3})}\\
 & \le C_{0}^{'}\sqrt{\frac{n\log n}{q}}\sum_{j=1}^{j_{0}}\epsilon^{j}\|UV^{*}\|_{\infty}
 &&\text{by (\ref{wlp2})}\\
 & \le C_{0}^{'}\sqrt{\frac{n\log n}{q}}\frac{1}{1-\epsilon}\|UV^{*}\|_{\infty}
 &&\text{by bound on geometric series}\\
 & \le C_{0}^{'}\sqrt{\frac{n\log n}{q}}\frac{1}{1-\epsilon}\frac{\sqrt{\mu r}}{n}
 &&\text{by RPCA assumptions}\\
 & \le C^{''}\epsilon<\frac{1}{4}
 &&\text{for some constant }C^{''}
\end{aligned}
\]

2) Proof of condition (2) : First, we expand,
\begin{eqnarray*}
\|P_{\Omega}(UV^{*}+W^{L})\|_{F} & = & \|P_{\Omega}(UV^{*}+P_{T^{\bot}}Y_{j_{0}})\|_{F}
\end{eqnarray*}

Then, because $P_{\Omega}(Y_{j_{0}})=P_{\Omega}(\sum_{j}P_{\Omega_{j}}Z_{j-1})=0$ and $P_{\Omega}(P_{T}(Y_{j_{0}})+P_{T^{\bot}}(Y_{j_{0}}))=0$, we have,
\begin{eqnarray*}
\|P_{\Omega}(UV^{*}+W^{L})\|_{F} & = & \|P_{\Omega}(UV^{*}-P_{T}Y_{j_{0}})\|_{F}
\end{eqnarray*}


Continuing,
\begin{align*}
\|P_{\Omega}(UV^{*}+W^{L})\|_{F}
& = \|P_{\Omega}(Z_{j_{0}})\|_{F}
&&\text{by definition}\\
& \le \|Z_{j_{0}}\|_{F}
&&\text{because of summing over larger set} \\
& \le \epsilon^{j_{0}} \sqrt{r}
&&\text{by (\ref{wlp3})} \\
& \le \sqrt{r}\frac{1}{n^{2}} \le \frac{\lambda}{4}
&&\text{by the choice of } \lambda \text{ and } \epsilon
\end{align*}


3) Proof of condition (3) :
\begin{align*}
\|P_{\Omega^{\bot}}(UV^{*}+W^{L})\|_{\infty}
& = \|P_{\Omega^{\bot}}(Z_{j_{0}}+Y_{j_{0}})\|_{\infty}
&&\text{by definition}\\
& \le \|Z_{j_{0}}\|_{\infty} + \|Y_{j_{0}}\|_{\infty}
&&\text{by triangle inequality and summing over larger set}\\
& \le \|Z_{j_{0}}\|_{F} + \|Y_{j_{0}}\|_{\infty}
&&\text{by the properties of Frobenius and infinite norms}\\
& \le \frac{\lambda}{8} + \|Y_{j_{0}}\|_{\infty}
&&\text{similar argument as in Proof of condition (2)}
\end{align*}


Moreover, we have
\[
\begin{aligned}
\|Y_{j_{0}}\|_{\infty}
& \le q^{-1}\sum_{j}\|P_{\Omega_{j}}Z_{j-1}\|_{\infty}
&&\text{ by triangle inequality}\\
& \le q^{-1}\sum_{j}\|Z_{j-1}\|_{\infty}
&&\text{ by summing over a larger set}\\
& \le q^{-1}\sum_{j}\epsilon^{j}\frac{\sqrt{\mu r}}{n}
&&\text{ by (\ref{wlp2})}\\
& \le \frac{\lambda}{8}
&&\text{if } \epsilon \text{ is sufficiently small}
\end{aligned}
\]
\end{proof}


%-----------------------------------------------------------------------------------------------------------------------------------------------------------------
\subsection{Proof of the Lemma about least square construction and dual certificate }

Construction of $W^{S}$:
%
\[
W^{S}=\lambda P_{T^{\bot}}((P_{\Omega}-P_{\Omega}P_{T}P_{\Omega})^{-1}sign(S_{0}))
\]

Now we present the proof of Lemma (\ref{ws}).
\begin{proof}
We consider the sign of $S_{0}$ to be distrbuted as follows
\[
sign(S_{0})_{i,j}=\begin{cases}
1 & \text{wp }\frac{\rho}{2}\\
0 & \text{wp }1-\rho\\
-1 & \text{wp }\frac{\rho}{2}
\end{cases}
\]


1) Proof of condition (1) :

I) We note the we can separate $W^{S}$into two parts and then bound them separately.
%
\begin{eqnarray*}
W^{S} & = & \lambda P_{T^{\bot}}(sign(S_{0}))+\lambda P_{T^{\bot}}(\sum_{k\ge1}(P_{\Omega}P_{T}P_{\Omega})^{k}(sign(S_{0})))
\end{eqnarray*}


II) Then, we have
\begin{align*}
\lambda \| P_{T^{\bot}}(sign(S_{0})) \|
& \le \lambda\| \sign(S_{0}) \|  &&\text{ by (\ref{property: p2})}\\
& = \frac{1}{\sqrt{n}}\|sign(S_{0})\| &&\text{ by the choice of }  \lambda\\
& \le 4\sqrt{\rho} &&\text{with high probability}
\end{align*}


where the last inequality uses the fact that for the entry-wise distribution of $sign(S_{0}$) , we can have $\|sign(S_{0})\|\le4\sqrt{n\rho}$ holds with high probability.

III) Now, for the other part, $\lambda P_{T^{\bot}}(\sum_{k\ge1}(P_{\Omega}P_{T}P_{\Omega})^{k}(sign(S_{0})))$, we bound it by first expressing it in the form of $<X,sign(S_{0})\rangle$ and then claim that with high probability, this term is bounded above as desired. Let $R=\sum_{k\ge1}(P_{\Omega}P_{T}P_{\Omega})^{k}$ ,
then we have,
\begin{eqnarray*}
\|P_{T^{\bot}}(R(sign(S_{0})))\| & \le & \|R(sign(S_{0}))\|\\
 & \le & 4\sup_{x,y\in N}<y,R(sign(S_{0})x)\rangle
\end{eqnarray*}


where the last inequality uses the fact that there exists a $\frac{1}{2}-net$ of the Eucledean ball and it has at most $6^{n}$ elements. Continuing, we have
\begin{align}
\|P_{T^{\bot}}(R(sign(S_{0})))\|
& \le 4\sup_{x,y \in N} \langle y,R(sign(S_{0})x) \rangle \nonumber \\
& = 4\sup_{x,y \in N} \langle yx^{*},R(sign(S_{0})) \rangle \nonumber \\
& = 4\sup_{x,y \in N} \langle R(yx^{*}),sign(S_{0}) \rangle
\label{net}
\end{align}


and that we denote $X(x,y)= \langle R(yx^{*}),sign(S_{0}) \rangle$ afterwards.

Note that, by Hoeffding's inequality, we have,
\begin{eqnarray*}
Pr(|X(x,y)|>t\mid\Omega) & \le & 2exp(-\frac{t^{2}}{2\|R(xy^{*})\|_{F}^{2}})
\end{eqnarray*}


This gives,
\begin{align*}
Pr(\|P_{T^{\bot}}(R(sign(S_{0})))\| > 4t\mid\Omega)
& \le Pr(\|R(sign(S_{0}))\| > 4t\mid\Omega)\\
& \le Pr(\sup_{x,y}|X(x,y)|>t\mid\Omega)
&& \text{ by (\ref{net})}\\
& \le 2N^{2}\exp \left( -\frac{t^{2}}{2\|R\|_{F}^{2}} \right)
&& \text{since } \|yx^{*}\|_{F} \le 1
\end{align*}


Now, we proceed to bound the probability without the condition on $\Omega$.

First, note that the event of $\|P_{\Omega}P_{T}\|\le\sigma=\rho+\epsilon$, implies that $\|R\|\le(\frac{\sigma^{2}}{1-\sigma^{2}})^{2}$. Thus, unconditionally, we have
\begin{eqnarray*}
Pr(\|R(sign(S_{0}))\|>4t) & \le & 2|N|^{2} \exp \left( \frac{-t^{2}}{2(\frac{\sigma^{2}}{1-\sigma^{2}})^{2}} \right) + Pr(\|P_{\Omega}P_{T}\|>\sigma)\\
 & \le & 2\cdot6^{2n} \exp \left( \frac{-t^{2}}{2(\frac{\sigma^{2}}{1-\sigma^{2}})^{2}} \right) + Pr(\|P_{\Omega}P_{T}\|>\sigma)
\end{eqnarray*}


Thus, where we finally put $t=\frac{1}{16}$
\begin{eqnarray*}
Pr(\lambda\|R(sign(S_{0}))\|>4t) & \le & 2\cdot6^{2n} \exp \left( \frac{-\frac{t^{2}}{\lambda^{2}}}{2(\frac{\sigma^{2}}{1-\sigma^{2}})^{2}} \right) + Pr(\|P_{\Omega}P_{T}\| > \sigma)
\end{eqnarray*}


With $\lambda=\sqrt{\frac{1}{n}},$we have this probability$\to0$ as $n\to\infty$. Thus with high probability $\|W^{S}\|\le\frac{1}{4}$

2) Proof of condition (2) :

The idea is that we first express $P_{\Omega^{\bot}}(W^{S})$ in the form of $<X,sign(S_{0})>$and we can derive upper bound on it if highly probably event of $\{\|P_{\Omega}P_{T}\|\le\sigma\}$ for some small $\sigma=\rho+\epsilon$ holds .

I) First,
\begin{align}
P_{\Omega^{\bot}}(W^{S})
& = P_{\Omega^{\bot}} \left( \lambda(I-P_{T}) \left( \sum_{k\ge0}(P_{\Omega}P_{T}P_{\Omega})^{k} \right) \sign(S_{0}) \right)
&& \text{since } P_{T^{\bot}} = I- P_T\\
& = -\lambda P_{\Omega^{\bot}} P_{T}(P_{\Omega}-P_{\Omega}P_{T}P_{\Omega})^{-1} \sign(S_{0})
&&\text{by  summing over terms and canceling}
\label{wsp1}
\end{align}


For $(i,j)\in\Omega^{C}$, we have
%
\[
\begin{aligned}
e_{i}^{*}W^{S}e_{j}
& = \langle e_{i}e_{j}^{*},W_{S} \rangle
&&\text{ by property of trace}\\
& = \langle e_{i}e_{j}^{*},-\lambda P_{\Omega^{\bot}}P_{T}(P_{\Omega}-P_{\Omega}P_{T}P_{\Omega})^{-1} \sign(S_{0}) \rangle
&&\text{ by (\ref{wsp1})}\\
& = -\lambda \langle e_{i}e_{j}^{*},P_{T}(P_{\Omega}-P_{\Omega}P_{T}P_{\Omega})^{-1} \sign(S_{0}) \rangle
&&\text{ by rearranging terms}\\
& = -\lambda \langle e_{i}e_{j}^{*},P_{T}P_{\Omega}(P_{\Omega}-P_{\Omega}P_{T}P_{\Omega})^{-1} \sign(S_{0}) \rangle
&&\text{ by the property of the inverse }\\
& = -\lambda \langle e_{i}e_{j}^{*},P_{T}\sum_{k\ge0}(P_{\Omega}P_{T}P_{\Omega})^{k} \sign(S_{0}) \rangle
&&\text{ by infinite sum representation}
\end{aligned}
\]

Noting that $P_{\Omega},P_{T}$ are self-adjoint, thus, we have
\begin{eqnarray}
e_{i}^{*}W^{S}e_{j} & = & \lambda \langle-(P_{\Omega}-P_{\Omega}P_{T}P_{\Omega})^{-1}P_{\Omega}P_{T}(e_{i}e_{j}^{*}), \sign(S_{0}) \rangle
\label{wsp2}
\end{eqnarray}


where we now denote $X(i,j)=-(P_{\Omega}-P_{\Omega}P_{T}P_{\Omega})^{-1}P_{\Omega}P_{T}(e_{i}e_{j}^{*})$

II)We now consider, where we put $t=\frac{1}{4}$,
\[
\begin{aligned}
Pr(\|P_{\Omega^{\bot}}(W^{S})\|_{\infty}>t\lambda\mid\Omega)
& \le \sum_{(i,j)\in\Omega^{C}}Pr(|e_{i}^{*}W^{S}e_{j}|>t\lambda|\Omega)
&&\text{by union bound}\\
& \le n^{2}Pr(|e_{i}^{*}W^{S}e_{j}|>t\lambda|\Omega)\text{ for some (i,j)}
&&\text{by taking the maximum}\\
& = n^{2}Pr(| \langle X(i,j), \sign(S_{0}) \rangle|>t|\Omega)
&&\text{ by (\ref{wsp2})}\\
& \le 2n^{2}\exp \left( -\frac{2t^{2}}{4\|X(i,j)\|_{F}} \right)
&&\text{by Hoeffding's inequality}
\end{aligned}
\]

III) We then proceed to bound the $\|X(i,j)\|$. On the event of $\{\|P_{\Omega}P_{T}\|\le\sigma\}$,
we have
\[
\begin{aligned}
\|P_{\Omega}P_{T}(e_{i}e_{j}^{*})\|_{F}
& \le \|P_{\Omega}P_{T}\|\cdot\|P_{T}(e_{i}e_{j}^{*})\|_{F}
&&\text{by property of spectral norm}\\
& \le \sigma\sqrt{\frac{2\mu r}{n}}
&&\text{ by (\ref{property: p4}) and the bound on } \|P_{\Omega}P_{T}\|
\end{aligned}
\]


Moreover, we have
\[
\begin{aligned}
\|(P_{\Omega}-P_{\Omega}P_{T}P_{\Omega})^{-1}\|
& \le \sum_{k\ge0}\|(P_{\Omega}P_{T}P_{\Omega})^{k}\|
&&\text{by triangle inequality}\\
& \le \frac{1}{1-\sigma}
&&\text{by the bound on } \|P_{\Omega}P_{T}\|
\end{aligned}
\]

Finally, we have
\begin{eqnarray*}
\|X(i,j)\|_{F} & \le & 2\sigma^{2}\frac{\frac{\mu r}{n}}{(1-\sigma)^{2}}
\end{eqnarray*}


Combining, we have
\begin{eqnarray*}
Pr(\|P_{\Omega^{\bot}}W^{S}\|>t\lambda) & \le & 2n^{2} \exp \left( \frac{-t^{2}n(1-\sigma)^{2}}{4\sigma^{2}(\mu r)} \right)+Pr(\|P_{\Omega}P_{T}\|\ge\sigma)\\
 & \le & \epsilon\text{ if }\mu r<\rho_{r}^{'}\frac{n}{\log n}
\end{eqnarray*}

\end{proof}



\subsection{Proof of the equivalence of the Bernoulli sampling and uniform sampling model } \label{sub:bernoullisampling}
To complete the story about the equivalence of sampling model, we present theorem.
\begin{theorem}
Let $E$ be the event that the recovery of $(L_{0},S_{0})$ is exact
through the RPCA. Then, $\forall\epsilon>0$,
\end{theorem}

\begin{itemize}
\item With $\rho=\frac{m}{n^{2}}+\epsilon$, $E$ holds with high probability when the sparse matrix $S_{i,j}\sim Bern(\rho)$ iid $\Longrightarrow$$E$ holds with high probability when the sparse matrix $S\sim Uniform(m)$.
\item With $\rho=\frac{m}{n^{2}}-\epsilon$, $E$ holds with high probability when the sparse matrix $S\sim Uniform(m)$ $\Longrightarrow$ $E$ holds with high probability when the sparse matrix $S_{i,j}\sim Bern(\rho)$
iid
\end{itemize}
\begin{proof}
Let us use the notation of subscrpt to denote the underlying sampling process, e.g. ,$P_{B(\rho)}(E)$ and $P_{U(m)}(E)$ be the probability of success recovery using Bernoulli sampling and uniform sampling respectively. We then upper and lower bound the difference of $P_{B(\rho)}(E)-P_{U(m)}(E)$ and show that the difference goes to zero as the dimension of the matrix $n\to\infty$. \\

\begin{align*}
P_{B(\rho)}(E)
& = \sum_{i=0}^{n^{2}}P_{B(\rho)}(|\Omega|=i)P_{B(\rho)}(E\mid|\Omega|=i)\\
& = \sum_{i=0}^{n^{2}}P_{B(\rho)}(|\Omega|=i)P_{U(i)}(E)\\
& \le \sum_{i=0}^{m-1}P_{B(\rho)}(|\Omega|=i)+\sum_{i=m}^{n^{2}}P_{U(i)}(E)P_{B(\rho)}(|\Omega|=i)\\
& \le \sum_{i=0}^{m-1}P_{B(\rho)}(|\Omega|=i)+\sum_{i=m}^{n^{2}}P_{U(i)}(E)P_{B(\rho)}(|\Omega|=i)\\
& \le \sum_{i=0}^{m-1}P_{B(\rho)}(|\Omega|=i)+\sum_{i=m}^{n^{2}}P_{U(m)}(E)P_{B(\rho)}(|\Omega|=i)\\
& \le P_{B(\rho)}(|\Omega|<m)+P_{U(m)}(E)
\end{align*}


This gives, $P_{B(\rho)}(E)-P_{U(m)}(E)\le P_{B(\rho)}(|\Omega|<m)$. With $\rho=\frac{m}{n^{2}}+\epsilon$, by law of large number, when $n\to\infty$ we get, $P_{B(\rho)}(E)\le P_{U(m)}(E)$ .

On the other hand,
\begin{align*}
P_{B(\rho)}(E)
& \ge \sum_{i=0}^{m}P_{B(\rho)}(|\Omega|=i)P_{B(\rho)}(E\mid|\Omega|=i)\\
& \ge P_{U(m)}(E)\sum_{i=0}^{m}P_{B(\rho)}(|\Omega|=i)\\
& = P_{U(m)}(E)(1-P_{B(\rho)}(|\Omega|>m))\\
& \ge P_{U(m)}(E)-P_{B(\rho)}(|\Omega|>m)
\end{align*}


This gives, $P_{B(\rho)}(E)-P_{U(m)}\ge-P_{B(\rho)}(|\Omega|>m)$. With $\rho=\frac{m}{n^{2}}-\epsilon$, by law of large number, when $n\to\infty$ we get, $P_{B(\rho)}(E)\ge P_{U(m)}(E)$ .
\end{proof}


%-----------------------------------------------------------------------------------------------------------------------------------------------------------------
\subsection{Proof of the form of sub-differential of nuclear norm } \label{sub:nuclearnorm}
To complete the story on the structure of subdifferential of nuclear norm, we present the following justifications.\\

\begin{definition}
For matrix norms $\|\cdot\|$ which satisfy $\|UAV\|=\|A\|$ $\forall U,V$ being orthonormal, then they are called orthogonally invariant norm.\\

\begin{definition}
For orthogonally invariant norm $\|\cdot\|$ which is defined by its singular values $\|A\|=\phi(\vec{\sigma})$ where $\vec{\sigma}$ are the singular values of $A$, we call the function $\phi$ as a symmetric gauge function if it is a norm and it satisfies $\phi(\vec{\sigma})=\phi(\epsilon_{1}\sigma_{i_{1}},...,\epsilon_{n}\sigma_{i_{n}})$ for any permulation of $(i_{1},...,i_{n})$ of $(1,...,n)$ and $\epsilon_{i}=\pm1$.

\begin{fact}
\label{fact10}
For orthogonally invariant norm $\|\cdot\|$ with symmetric
gauge function $\phi$, the sub-differential is given by

\begin{eqnarray*}
\partial\|A\| & = & \{Udiag(\vec{d})V\mid A=U\Sigma V^{T},\vec{d}\in\partial\phi(\vec{d}),U\in R^{m},V\in R^{n}\}
\end{eqnarray*}

\end{fact}
\end{definition}
\end{definition}

\begin{thm}
Let $A=U^{(1)}\Sigma{V^{(1)}}^{T}$. Then
\[
\partial\|A\|_{*}=\{U^{(1)}{V^{(1)}}^{T}+W:\|W\|\le1,{U^{(1)}}^{T}W=0,WV^{(1)}=0\}
\]
\end{thm}

\begin{proof}
We take the symmetric gauge function as $\|\cdot\|_{1}$ and then apply the Fact (\ref{fact10}) and will obtain the desired result.
\end{proof}



%============================================================================================
\newpage
\section{Related Problems and Extensions}


%-----------------------------------------------------------------------------------------------------------------------------------------------------------------
\subsection{Exact Matrix completion}
Robust PCA is an extension of the exact matrix completion problem introduced in \cite{Candes:2009uq}, where one seeks to recover a low-rank matrix $L_0$ from a small fraction of its entries. More precisely, assume one is given $\{(L_0)_{ij}, (i,j)\in \Omega\}$ where $\Omega$ is a subset of $[n]\times [n]$.

Problem to solve

\begin{equation}
\begin{aligned}
&\text{minimize} && \text{rank}(L) \\
&\text{subject to} && P_\Omega L = P_\Omega L_0
\end{aligned}
\end{equation}


A heuristic is to minimize the nuclear norm of $L$

%-------------------------------------------------------------------------
\subsubsection{Incoherence}

Singular vectors have to be sufficiently spread

\begin{equation}
\mu(U) = \frac{n}{r} \max_i \|P_U e_i\|_2^2 = \frac{n}{r} \max_i \left[ \sum_{k=1}^r u_{ki}^2 \right]
\label{emc_incoherence}
\end{equation}

Assumptions:
\begin{itemize}
\item $\max \{\mu(U), \mu(V)\} \leq \mu_0$
\item $\left( \sum_k u_kv_k^*\right)_{ij} \leq \mu_1 \sqrt{\frac{r}{n_1 n_2}}$ (true for $\mu_1 = \mu_0\sqrt{r}$)
\item $m \geq c \max \{ \mu_1^*, \sqrt{\mu_0}\mu_1, \mu_0 n^{1/4}\}n r \beta \log n$
\end{itemize}

Under these assumptions, recovery is exact with high probability (at least $1-\frac{c}{n\beta}$)

Incoherent matrices:
\begin{itemize}
\item sampled from the incoherent basis model
\item sampled from the random orthogonal model: if $M = \sum_k \sigma_k u_k v_k^*$, then $\{u_1, \dots, u_r\}$ and $\{v_1, \dots, v_r\}$ are assumed to be selected at random.

\end{itemize}

%-------------------------------------------------------------------------
\subsubsection{Main result}




%-------------------------------------------------------------------------
\subsubsection{Comparing results to Robust PCA}
Robust PCA can be thought of as an extension of the matrix completion problem, where instead of having a known subset of the entries $\{(L_0)_{ij}, (i,j)\in \Omega\}$ and the rest is missing, we have an unknown subset of the entries and the rest is corrupted. In this sense, Robust PCA is a harder problem.

Note that the matrix $L_0$ can be recovered by Principal Component Pursuit, solving a different problem:
\begin{equation}
\begin{aligned}
&\text{minimize} && \|L\|_* + \lambda \|S\|_1\\
&\text{subject to} && P_\Omega (L+S) = P_\Omega L_0
\end{aligned}
\end{equation}


%TODO: numerical simulations



%-----------------------------------------------------------------------------------------------------------------------------------------------------------------
\subsection{Stable Principal Component Pursuit}
\subsubsection{Overview}

The paper studies the problem of recovering a low-rank matrix (the principal components) from a high- dimensional data matrix despite both small entry-wise noise and gross sparse errors. It proves that the solution to a convex program (a relaxation of classic Robust PCA) gives an estimate of the low-rank matrix that is simultaneously stable to small entry- wise noise and robust to gross sparse errors. The result shows that the proposed convex program recovers the low-rank matrix even though a positive fraction of its entries are arbitrarily corrupted, with an error bound proportional to the noise level. 

%-----------------------------------------------------------------------------
\subsubsection{Main result}

The paper consider a matrix $M\in\mathbb{R}^{n_1\times n_2}$ of the from $M = L_0+S_0+Z_0$, where~$L_0$ is (non-sparse) low rank, $S_0$ is sparse (modeling gross errors) and $Z_0$ is ``small'' (modeling a small noisy perturbation). The assumption on~$Z_0$ is simply that $||Z_0||_F \leq \delta$ for some small known~$\delta$. Hence at least for the theory part of the paper the authors do not assume anything about the distribution of the noise other than it is bounded (however they will gloss over this in their algorithm). 

The convex program to be solved is a slight modification of the standard Robust PCA problem and given by
\begin{align}
\begin{split}
\min_{L,S} \; &||L||_* + \lambda ||S||_1 \\
\text{s.t.} \quad &||M-L-S||_F \leq \delta
\end{split}
\label{mainresult:optproblem}
\end{align}
where $\lambda = 1/\sqrt{n_1}$. Under a standard incoherence assumption on~$L_0$ (which essentially means that $L_0$ should not be sparse) and a uniformity assumption on the sparsity pattern of~$S_0$ (which means that the support of~$S_0$ should not be too concentrated) the main result states that, with high probability in the support of~$S_0$, for any~$Z_0$ with $||Z_0||_F \leq \delta$, the solution $(\hat{L},\hat{S})$ to~\eqref{mainresult:optproblem} satisfies
\begin{align*}
||\hat{L}-L_0||_F^2 + ||\hat{S}-S_0||_F^2 \leq C n_1n_2\delta^2
\end{align*}
where~$C$ is a numerical constant. The above claim essentially states that the recovered low-rank matrix~$\hat{L}$ is stable with respect to non-sparse but small noise acting on all entries of the matrix.

In order to experimentally verify the predicted performance to their formulation, the author provide a comparison with an oracle. This oracle is assumed to provide information about the support of~$S_0$ and the row and column spaces of~$L_0$, which allows the computation of the MMSE estimator which otherwise would be computationally intractable (strictly speaking it of course is not really the MMSE, since it uses additional information from the oracle). Simulation results that show that the RMS error of the solution obtained through~\eqref{mainresult:optproblem} in the non-breakdown regime (that is, for the support of~$S_0$ sufficiently small) is only about twice as large as that of the oracle-based MMSE. This suggests that the proposed algorithm works quite well in practice. 



%-----------------------------------------------------------------------------
\subsubsection{Relations to existing work}

The result of the paper can be seen from two different view points. On the one hand, it can be interpreted from the point of view of standard PCA. In this case, the result states that standard PCA, which can in fact be shown to be statistically optimal w.r.t. i.i.d Gaussian perturbations, can also be made robust with respect to sparse gross corruptions. On the other hand, the result can be interpreted from the point of view of Robust PCA. In this case, it essentially states that the classic Robust PCA solution can itself be made robust with respect to some small but non-sparse noise acting on all entries of the matrix. 

Conceptually, the work presented in the paper is similar to the development of results for ``imperfect'' scenarios in compressive sensing where the measurements are noisy and the signal is not exact sparse. In this body of literature, $l_1$-norm minimization techniques are adapted to recover a vector $x_0 \in \mathbb{R}^n$ from contaminated observations $y=Ax_0+z$, where $A \in \mathbb{R}^{m\times n}$ with $m \ll n$ and z is the noise term. 


%-----------------------------------------------------------------------------
\subsubsection{Algorithm}

For the case of a noise matrix~$Z_0$ whose entries are i.i.d. $\Ncal(0,\sigma^2)$, the paper suggests to use an Accelerated Proximal Gradient (APG) algorithm (see algorithms section for details) for solving~\eqref{mainresult:optproblem}. Note that for~$\delta =0$ the problem reduces to the standard Robust PCA problem with an equality constraint on the matrices. For this case the APG algorithm proposed in~\cite{Lin:2009kx} solves an approximation of the form
\begin{align*}
\min_{L,S} \; &||L||_* + \lambda ||S||_1 + \frac{1}{2\mu} ||M-L-S||_F
\end{align*}
For the Stable PCP problem where~$\delta>0$ the authors advocate using the same algorithm with fixed but carefully chosen parameter~$\mu$ (similar to~\cite{Candes:2010fk}). In particular, they point out\footnote{this based on the strong Bai Yin Theorem~\cite{Bai:1988fk}, which implies that for an $n\times n$ real matrix with entries $\xi_{ij} \sim \Ncal(0,1)$ the it holds that $\limsup_{n\rightarrow \infty} \norm{Z_0}{2}{}/\sqrt{n} = 2$ almost surely}
 that for $Z_0 \in\Rbf^{n\times n}$ with~$(Z_0)_{ij} \sim \Ncal(0,\sigma^2)$ i.i.d. it holds that $n^{-1/2}\norm{Z_0}{2}{} \rightarrow \sqrt{2}\sigma$ almost surely as $n\rightarrow\infty$. They then choose the parameter~$\mu$ such that if $M=Z_0$, i.e. if $L_0=S_0=0$, the minimizer of the above problem is likely to be $\hat{L}=\hat{S}=0$. The claim is that this is the case for~$\mu = \sqrt{2n}\sigma$.
  
It is worth noting that the assumption of a Gaussian noise matrix~$Z_0$ is reasonable but not always satisfied. If it is not, then it is not clear if using the APG algorithm to solve the associated approximate problem is a good idea and different algorithms may be needed. The problem~\eqref{mainresult:optproblem} can be expressed as an SDP and can therefore in principle be solved using general purpose interior point solvers. However, the same scalability issues as in the standard Robust PCA problem will limit prohibit to use these methods for high-dimensional data. The paper~\cite{Aybat:2011vn} focuses on efficient first-order algorithms for solving~\eqref{mainresult:optproblem}.


%-----------------------------------------------------------------------------
\subsubsection{Conclusion and Outlook}

The paper addresses a problem of potentially very high practical relevance. While it is reasonable to assume that in many applications the low-rank component~$L_0$ will only be corrupted by a comparatively small number of gross errors (caused by rare and isolated events), the assumption of perfect measurements for the rest of the data outside the support of~$S_0$ that is made in classic Robust PCA will generally not hold for example due to sensor noise. This paper asserts that if the non-sparse noise component~$Z_0$ is sparse, then with high probability the recovered components are ``close'' to the actual ones. 

For simplicity, the paper models the non-sparse noise simply as an additive perturbation that is bounded in the Frobenius norm. In cases where one has additional information available about this noise, for example its distribution or some bounds on the absolute value of each entry, it might be possible to derive better bounds on the resulting errors. One possible extension could therefore be to look at exploiting structure in the noise. 

One thing the paper claims is that ``at a cost not so much higher than the classical PCA, [the] result is expected to have significant impact on many practical problems''. As mentioned above I do agree that the result has a significant impact on many practical problems. However, the claim concerning the computational complexity is very optimistic. The fastest solver for the special case $\delta =0$ (classic Robust PCA) currently seems to be a alternating directions augmented Lagrangian method. This method requires an SVD at each iteration, and for problems involving large-scale data the number of iterations can be very large. The standard PCP algorithm on the other hand is based on a single SVD, hence it can be computed much faster. 


%-----------------------------------------------------------------------------------------------------------------------------------------------------------------
\subsection{Robust Alignment by Sparse and Low-rank Decomposition}
\label{subsec: RASL}

The convex optimization framework for low-rank matrix recovery has been employed successfully. However, in practice, much more data can be viewed as low-rank only after some transformation is applied. The new formulation of this problem as Robust Alignment by Sparse and Low-rank Decomposition (RASL) \cite{Peng:2010}:
\begin{align}
\min_{A, E, \tau}  ||A||_{*} + \lambda||E||_{1} \quad  \text{s.t.} \;  D\circ\tau = A+E
\label{eq:rasl:original}
\end{align}
where $A  \in\mathbb{R}^{m\times n}$ is low-rank matrix, $A\in\mathbb{R}^{m\times n}$ is sparse matrix, $D$ is our measurements, which is the result of $(A+E)$ subjecting to transformation $\tau^{-1}$. Here we assume that the transformation is invertible. 
We define $D\circ\tau$ as: $D\circ\tau = [\;D_{1}\circ\tau_{1} \;|\;D_{2}\circ\tau_{2} \;|\; \dots \;|\; D_{n}\circ\tau_{n}\;]$, which is the measurements $D=[\;D_{1} \;|\;D_{2} \;|\; \dots \;|\; D_{n}\;] $subjects to set of transformations $\tau=[\;\tau_{1} \;|\;\tau_{2} \;|\; \dots \;|\; \tau_{n}\;] \in\mathbb{G}^n$, where $\mathbb{G}$ is a group of certain type of invertible transformations, which could be affine transform, rotation transform, etc.  \\

The main difficulty in solving ~\eqref{eq:rasl:original} is the nonlinearity of constraint $D\circ\tau = A+E$. When the change in $\tau$ is small, we can approximate this constraint by linearizing about the current estimate of $\tau$. Here, we assume that $\mathbb{G}$ is some $p$-parameter group and identify $\tau=[\;\tau_{1} \;|\;\tau_{2} \;|\; \dots \;|\; \tau_{n}\;] \in \mathbb{R}^{p\times n}$ with the parameterizations of all of the transformations. For $\Delta\tau = [\;\Delta\tau_{1} \;|\; \Delta\tau_{2} \;|\; \dots \;|\; \Delta\tau_{n}\;]$, write $D\circ(\tau+\Delta\tau) \approx D\circ\tau + \sum_{i=1}^n J_{i}\Delta\tau_{i}\epsilon_{i}$, where $J_{i} \doteq \frac{\partial}{\partial\zeta}(D_{i}\circ\zeta)|_{\zeta = \tau_{i}}$ is the Jacobian of the $i$-th measurement with respect to the transformation parameters $\tau_{i}$. $\{\epsilon_{i}\}$ denotes the standard basis for $\mathbb{R}^n$. This leads to a convex optimization problem in unknowns $A, E, \Delta\tau$:
\begin{align}
\min_{A, E, \Delta\tau}  ||A||_{*} + \lambda||E||_{1}  \quad \text{s.t.} \;  D\circ\tau + \sum_{i=1}^n J_{i}\Delta\tau\epsilon_{i}\epsilon_{i}^{T}= A+E
\label{eq:rasl:linearized}
\end{align}
It leads to algorithm~\ref{alg:RASL}




\begin{algorithm}[h!]
\label{alg:RASL}
\caption{RASL}
\KwIn{$D = [\; D_{1} \;|\; D_{2} \;|\; \dots \;|\; D_{n}]$, initial transformation $\tau_{1}, \tau_{2}, \dots, \tau_{n}$ in a certain parametric group $\mathbb{G}$, weight $\lambda > 0$.}
\While{not converged}{
\textbf{Step 1:} compute Jacobian matrices w.r.t. transformation: %$J_{i} \leftarrow \frac{\partial}{\partial\zeta}(D_{i}\circ\zeta)|_{\zeta = \tau_{i}}$ \\
\begin{equation}
J_{i} \leftarrow \frac{\partial}{\partial\zeta}(D_{i}\circ\zeta)|_{\zeta = \tau_{i}} \nonumber
\end{equation}
\textbf{Step 2 (inner loop):} solve the linearized convex optimization:
\begin{equation}
(A^{*}, E^{*}, \Delta\tau^{*}) \leftarrow \argmin_{A, E, \Delta\tau}  ||A||_{*} + \lambda||E||_{1}  \quad \text{s.t.} \;  D\circ\tau + \sum_{i=1}^n J_{i}\Delta\tau\epsilon_{i}\epsilon_{i}^{T}= A+E  \nonumber
\end{equation}
\textbf{Step 3:} update the transformation: $\tau \leftarrow \tau + \Delta\tau^{*}$
\newline
}
\KwOut{$A^{*}, E^{*}, \tau^{*}$}
\end{algorithm}

%-----------------------------------------------------------------------------------------------------------------------------------------------------------------
\subsection{Robust Matrix Decomposition With Sparse Corruptions: D Hsu et. al.}

%------------------------------------------------------------------------------
\subsubsection{Question being addressed}

Under deterministic setting, it studies how much sparsity is allowed for accurate recovery of the sparse-lowrank pairs.


%------------------------------------------------------------------------------
\subsubsection{Main ideas}

Given $M$ as the observed matrix, it analyze the following two optimization problems. With arguments as $(L,S)$,

\begin{eqnarray}
\min_{(L,S)} & ||L||_{*}+\lambda||S||_{1}\nonumber \\
s.t. & ||L+S-M||_{1}\le\epsilon_{1}\\
 & ||L+S-M||_{*}\le\epsilon_{*}\nonumber 
\end{eqnarray}


and 
\begin{eqnarray}
\min_{(L,S)} & ||L||_{*}+\lambda||S||_{1}+\frac{1}{2\mu}||L+S-M||_{F}
\end{eqnarray}


It is remarked that the $M$ is a perturbed observation outcome of the original $(L_{0},S_{0})$ pairs.


%------------------------------------------------------------------------------
\subsubsection{Contributions}
\begin{enumerate}
\item It provides sufficient conditions on sparsity of the original $(L_{0},S_{0})$ pairs that allow accurate recovery in the sense that $||L_{0}-\hat{L}||_{\infty},||S_{0}-\hat{S}||_{\infty}$ is small.
\item If the observed matrix $M$ is pertured from $L_{0}+S_{0}$ by a small amount (i.e.$\epsilon$), the optimizer $(\hat{L}, \hat{S})$ will be $\epsilon-$close to the orginal $(L_{0},S_{0})$ pairs.
\end{enumerate}

%-----------------------------------------------------------------------------------------------------------------------------------------------------------------
\subsection{Characterization of the subdifferential of some matrix norms : GA Watson et. al.}


%------------------------------------------------------------------------------
\subsubsection{Significance}

Subdifferential of nuclear norm is a critical element in the formulation of dual certificates. This paper proves why the subdifferential of nuclear norm is of the form.

\[
\{U^{(1)}V^{(1)T}+W|U^{(1)T}W=0,WV^{(1)},||W||_{2}\le1\}
\]


where $A=[U^{(1)}|U^{(2)}]
[
\begin{array}{cc}
\Sigma & 0\\
0 & 0
\end{array}
]
[
\begin{array}{c}
V^{(1)T}\\
V^{(2)T}
\end{array}
]
$


%------------------------------------------------------------------------------
\subsubsection{Details}

\begin{lemma}
If a norm $||\cdot||$ is orthogonal invariant and is characterized by a symmetric gauge function, then
\textup{
\[
\partial||A||=conv\{Udiag(\vec{d})V|A=U[\begin{array}{cc}
\Sigma & 0\\
0 & 0
\end{array}]V^{T},\vec{d}\in\partial\phi(\Sigma)\}
\]
}
\end{lemma}

\begin{proof}
Sketch: For $RHS\subset LHS$, one can check that they satisfy the dual norm characterization of $\partial||A||$. For $LHS\subset RHS$, one can use maximum representation of the directional derivative of
the $||A||$ to show that subdifferential not inside RHS will give
a contradiction. 
\end{proof}

\begin{prop}
\[
\partial||A||_{*}=\{U^{(1)}V^{(1)T}+W|U^{(1)T}W=0,WV^{(1)},||W||_{2}\le1\}
\]
\end{prop}

\begin{proof}
Sketch: Take $\phi(\vec{\sigma}) =  \|\sigma \|_1$ and apply Lemma~\ref{} to the convex hull of all SVD of $A$.
\end{proof}



%============================================================================================
\section{Generalization of the Robust PCA scheme with known rank}


%-----------------------------------------------------------------------------------------------------------------------------------------------------------------
\subsection{Introduction }

Recall that the nuclear norm introduced in the PCP scheme is resulted because we would like to extract the low rank component from gross random noise. Nuclear norm is used because it is a heuristic for penalizing high rank matrix. Now, we consider the case when we have extra information/guess from the data set that we know precisely what the rank of the matrix are. Therefore, it is natural to introduce the following the following heuristics. 

\begin{eqnarray*}
E^{*} & = & \min_{\{p_{i}\}\{q_{j}\},||p_{j}||_{2},1\le j\le r}||M-\sum_{j=1}^{r}p_{j}q_{j}^{T}||_{1}
\end{eqnarray*}


We specialize the discussion to the case of rank-1 approximation (when $r=1$) and shows that this scheme has a  probabilistic gurantee of recovery and demonstrate that an efficient power iteration can be employed to solve this problem. Empirical results are also provided to support this method. We remark that we can generalize this and obtain similar results. 


%-----------------------------------------------------------------------------------------------------------------------------------------------------------------
\subsection{Problem statement.}

We are now interested in solving the following problem. 

\begin{eqnarray}
E^{*} & = & \min_{p,q}||M-pq^{T}||_{1}\label{rankonemin}
\end{eqnarray}



%-----------------------------------------------------------------------------------------------------------------------------------------------------------------
\subsection{Algorithm derivation}

Let $M=(a_{i,j})\in R^{mXn}$. We now employ the block-coordinate descent method to solve this problem. Note that 

\begin{eqnarray}
\min_{p} & ||M-pq^{T}|| & =\sum_{i=1}^{m}\min_{t}(\sum_{j=1}^{n}|a_{i,j}-tq_{j}|)\\
 &  & =\sum_{i=1}^{m}\min_{t}(\sum_{j=1}^{n}|q_{j}||t-\frac{a_{i,j}}{q_{j}}|)\nonumber 
\end{eqnarray}
\begin{eqnarray}
\min_{q} & ||M-pq^{T}|| & =\sum_{j=1}^{n}\min_{t}(\sum_{i=1}^{m}|a_{i,j}-tp_{i}|)\\
 &  & =\sum_{j=1}^{n}\min_{t}(\sum_{j=1}^{n}|p_{i}||t-\frac{a_{i,j}}{p_{i}}|)\nonumber 
\end{eqnarray}


And for solving the subproblem of finding 
\[
\min_{t}\sum_{k=1}^{k_{0}}c_{i}|t-d_{i}|
\]
where $c_{i}\ge0$ is basically finding the weighted median and can be done by the following method with complexity $O(k_{0}\log k_{0})$ mostly on sorting the sequence. We call it WMH.

\begin{algorithm}[tbh]
\begin{enumerate}
\item We first sort $\vec{\ensuremath{d}}$ s.t. $d_{i_{1}}\le d_{i_{2}}\le...\le d_{i_{k_{0}}}$ 
\item We then find $k^{'}$s.t. 
\begin{eqnarray*}
\sum_{\theta=1}^{k^{'}-1}c_{i_{\theta}} & \le & \sum_{\theta=k^{'}}^{k_{0}}c_{i_{\theta}}\\
\sum_{i=1}^{k^{'}}c_{i_{\theta}} & \ge & \sum_{i=k^{'}+1}^{k_{0}}c_{i_{\theta}}
\end{eqnarray*}

\item We then set $t^{*}$to be $d_{i_{k^{'}}}$
\end{enumerate}
\caption{WMH $(k_{0},\vec{c},\vec{d})$}
\end{algorithm}


This algorithm is optimal in finding $t$. This is justified by using the property of sub-differential of $||\cdot||_{1}$ and note that $0\in\partial(\sum_{k=1}^{k_{0}}c_{i}|t^{*}-d_{i}|)$. 

Now we are ready to state the power iteration method to solve the optimization rank-1 optimization problem. We call it Poweriteration.

\begin{algorithm}[tbh]
Repeat
\begin{enumerate}
\item $p_{i}\leftarrow wmh(n,abs(q),M(i,:)./q)$ for each i
\item $p\leftarrow\frac{p}{max(p)}$
\item $q_{j}\leftarrow wmh(n,abs(p),M(:,j)./p)$ for each j
\end{enumerate}
Until stopping criterion is met

\caption{Poweriteration($M$)}
\end{algorithm}



%-----------------------------------------------------------------------------------------------------------------------------------------------------------------
\subsection{Performance Guarantee}

Here are two results about performance guarantee. One is a result for deterministic case and the other is for the random case. 
\begin{prop}
Let $M=pq^{T}+S$ with $\frac{2}{\epsilon}||S||_{1}<||pq^{T}||_{1}$. Then the $(\hat{p},\hat{q})$ recovered from (\ref{rankonemin})
satisfy 
\begin{eqnarray*}
\frac{||pq^{T}-\hat{p}\hat{q}^{T}||}{||pq^{T}||_{1}} & \le & \epsilon
\end{eqnarray*}
\end{prop}
\begin{proof}
If $(\hat{p},\hat{q})=(p,q)$ then the proposition is trivially true.
Now assume $(\hat{p},\hat{q})\not=(p,q)$. Then we have ,

\begin{eqnarray*}
||S||_{1} & \ge & ||pq^{T}+S-\hat{p}\hat{q}^{T}||_{1}\\
 & \ge & ||pq^{T}-\hat{p}\hat{q}^{T}||_{1}-||S||_{1}
\end{eqnarray*}


Now if the conclusion is false, we will have 
\begin{eqnarray*}
||S||_{1} & \ge & \epsilon||pq^{T}||_{1}-||S||_{1}
\end{eqnarray*}


This gives a contradiction of, 
\begin{eqnarray*}
2||S||_{1} & \ge & \epsilon||pq^{T}||_{1}
\end{eqnarray*}
\end{proof}
\begin{prop}
Let $M=pq^{T}+S$ be $R^{nXn}$ and $S$ has $k$ possibly non-zero
elements. If $S_{i}\sim U[-x_{s},x_{s}]$, $p_{i}\sim U[-x_{p},x_{p}]$,$q_{i}\sim U[-x_{p},x_{p}]$
independently distributed, with $\frac{n}{k}\ge\frac{8x_{s}}{\epsilon x_{p}x_{q}}$,
then 
\begin{eqnarray*}
\lim_{n\to\infty}Pr(\frac{||pq^{T}-\hat{p}\hat{q}^{T}||}{||pq^{T}||_{1}}>\epsilon) & = & 0
\end{eqnarray*}
\end{prop}
\begin{proof}
Let the event that $\frac{||pq^{T}-\hat{p}\hat{q}^{T}||}{||pq^{T}||_{1}}>\epsilon$
be an error event. If error occurs, then 

\begin{eqnarray*}
kx_{s} & \ge & ||pq^{T}-\hat{p}\hat{q}^{T}+S||_{1}\\
 & \ge & ||pq^{T}-\hat{p}\hat{q}^{T}||_{1}-||S||_{1}\\
 & \ge & \epsilon||pq^{T}||_{1}-kx_{s}
\end{eqnarray*}


We now bounding the probability of error. 
\begin{eqnarray*}
 &  & Pr(\frac{||pq^{T}-\hat{p}\hat{q}^{T}||}{||pq^{T}||_{1}}>\epsilon)\\
 & \le & Pr(kx_{s}\ge\epsilon||pq^{T}||_{1}-kx_{s})\\
 & \le & Pr(\frac{2kx_{s}}{\epsilon}\ge||pq^{T}||_{1})\\
 & \le & (1-(\frac{t}{x_{q}})^{n})Pr(\frac{1}{n}\sum_{i=1}^{n}|p_{i}|\le\frac{2kx_{s}}{tn\epsilon})+(\frac{t}{x_{q}})^{n}
\end{eqnarray*}


Then by Chebyshev inequality, we have, by putting $t=\frac{1}{2x_{q}}$
\begin{eqnarray*}
 &  & Pr(\frac{||pq^{T}-\hat{p}\hat{q}^{T}||}{||pq^{T}||_{1}}>\epsilon)\\
 & \le & (1-(\frac{1}{2})^{n})\frac{\frac{1}{n}(\frac{x_{p}^{2}}{12})}{(\frac{x_{p}}{2}-\frac{2kx_{s}}{n\epsilon x_{q}})}+(\frac{1}{2})^{n}
\end{eqnarray*}


And this goes to zero as $n\to\infty$.
\end{proof}



%-----------------------------------------------------------------------------------------------------------------------------------------------------------------
\subsection{Empirical Evidence}

We simulate the power iteration in recovering sparse noise. In this simulation, we use randomly generated the entries of $p$ and $q$ as $N(0,1)$ iid. And then we randomly generate sparse matrix with sparse support uniformly distributed across the nXn matrix. And each sparse entry has a value with distribution of $N(0,1)$. We then plot the graph of different degree of sparsity and the corresponding effectiveness of the optimization heuristic in extracting the original $pq^{T}$.
The result is as follows.


\begin{figure}[tbh]
%\includegraphics[width=6in]{\string"C:/Users/kk/Desktop/EE227A/project/matlab experiments/fastalgo\string".jpg}
\caption{Simulation result}
\end{figure}






%============================================================================================
\section{Algorithms}


%============================================================================================
\section{Applications}





%============================================================================================
\newpage
\bibliographystyle{abbrv}
\bibliography{../common/RobustPCA}

\end{document}
% !TEX root = report.tex
\newpage
%-----------------------------------------------------------------------------------------------------------------------------------------------------------------
\subsection{Proof of the Lemma about golfing scheme and dual certificate }
Golfing scheme:

The golfing scheme involves creating a $W^{L}$ according to the following method.
\begin{enumerate}
\item Fix $j_{0}\ge1$, define $\Omega_{j}\sim Bern(q)$ iid with $1\le j\le j_{0}$ and $\rho=(1-q)^{j_{0}}$. Define the complement of support of $\Omega$ by $\Omega=\cup_{1\le j\le j_{0}}\Omega_{j}^{C}$.
\item Define a sequence of matrix which finally ends at $W^{L}$

\begin{enumerate}
\item $Y_{0}=0$
\item $Y_{j}=Y_{j-1}+\frac{1}{q}P_{\Omega_{j}}P_{T}(UV^{*}-Y_{j-1})$ for $1\le j\le j_{0}$
\item $W^{L}=P_{\ensuremath{T^{\bot}}}(Y_{j_{0}})$
\end{enumerate}

\end{enumerate}

We first list a number of facts that will be used in the proof of Lemma (\ref{wl}).

\begin{fact}
\label{fact2}
If we fix $Z\in T$, $\Omega_{0}\sim Bern(\rho_{0})$, and $\rho_{0}\ge C_{0}\epsilon^{-2}\frac{\mu r\log n}{n}$ , then with high probability, we will have,

\label{fact5}
\[
||Z-\rho_{0}^{-1}P_{T}P_{\Omega_{0}}(Z)||_{\infty}\le\epsilon||Z||_{\infty}
\]

\begin{fact}
\label{fact3}
If we fix $Z$, $\Omega_{0}\sim Bern(\rho_{0})$, and $\rho_{0}\ge C_{0}\frac{\mu\log n}{n}$, then with high probability,
we will have,
\begin{eqnarray*}
||(I-\rho_{0}^{-1}P_{\Omega_{0}})Z|| & \le & C_{0}^{'}\sqrt{\frac{n\log n}{\rho_{0}}}||Z||_{\infty}
\end{eqnarray*}

\begin{fact}
\label{fact4}
If $\Omega_{0}\sim Bern(\rho_{0})$,$\rho_{0}\ge C_{0}\epsilon^{-2}\frac{\mu r\log n}{n}$, then with high probability, we will have,

\[
||P_{T}-\rho_{0}^{-1}P_{T}P_{\Omega_{0}}P_{T}||\le\epsilon
\]

\begin{fact}
If $\Omega_{0}\sim Bern(\rho)$ and $1-\rho\ge C_{0}\epsilon^{-2}\frac{\mu r\log n}{n}$, then with high probability $||P_{\Omega}P_{T}||^{2}\le\rho+\epsilon$
\end{fact}
\end{fact}
\end{fact}
\end{fact}


Now we present the proof of Lemma (\ref{wl})

\begin{proof}
We define another sequence of matrix $Z_{j}=UV^{*}-P_{T}(Y_{j})$. There are some properties about $Z_{j}$ which allows us to establish the proof. We survey them here and provides the proof of them.

i) Note that 
\begin{equation}
Z_{j}=(P_{T}-\frac{1}{q}P_{T}P_{\Omega_{j}}P_{T})Z_{j-1}. \label{wlp1}
\end{equation}
The reason is as follows.
\begin{align*}
Z_{j}
 & = UV^{*}-P_{T}(Y_{j-1}+\frac{1}{q}P_{\Omega_{j}}P_{T}(UV^{*}-Y_{j-1})) &\text{by construction of } Y_j \\
 & = UV^{*}-P_{T}(Y_{j-1})-\frac{1}{q}P_{T}P_{\Omega_{j}}P_{T}(UV^{*}-Y_{j-1}) &\text{by of linearity of }P_{T}\\
 & = Z_{j-1}-q^{-1}(P_{T}P_{\Omega_{j}}(UV^{*}-P_{T}(Y_{j-1}))) &\text{ since }P_{T}(UV^{*})=UV^{*}\\
 & = P_{T}(Z_{j-1})-q^{-1}(P_{T}P_{\Omega_{j}}P_{T}Z_{j-1}) &\text{ since }Z_{j-1}\in T\\
 & = (P_{T}-q^{-1}P_{T}P_{\Omega_{j}}P_{T})Z_{j-1} &\text{by linearity }
\end{align*}


ii) If $q\ge C_{0}\epsilon^{-2}\frac{\mu r\log n}{n}$, then  with high probability, 
\begin{equation} 
||Z_{j}||_{\infty}\le\epsilon^{j}||UV^{*}||_{\infty} \label{wlp2}
\end{equation}
 The reason is as follows. By Fact (\ref{fact2}), we have,

\begin{eqnarray*}
||Z_{j-1}-q^{-1}P_{T}P_{\Omega_{j}}Z_{j-1}||_{\infty} & \le & \epsilon||Z_{j-1}||_{\infty} \\
||Z_{j}||_{\infty} & \le & \epsilon||Z_{j-1}||_{\infty} \text{because of (\ref{wlp1})}
\end{eqnarray*}


Inductively, we get desired.

iii) If $q\ge C_{0}\epsilon^{-2}\frac{\mu r\log n}{n}$, then 
\begin{equation} 
||Z_{j}||_{F}\le\epsilon^{j}\sqrt{r}  \label{wlp3}
\end{equation} 
The reason is as follows. By Fact (\ref{fact4}), we have,

\begin{eqnarray*}
||(P_{T}-q^{-1}P_{T}P_{\Omega_{0}}P_{T})(\frac{Z_{j-1}}{||Z_{j-1}||_{F}})||_{F} & \le & \epsilon\\
||(P_{T}-q^{-1}P_{T}P_{\Omega_{0}}P_{T})Z_{j-1}||_{F} & \le & \epsilon||Z_{j-1}||_{F} \text{by rearranging terms}\\
||Z_{j}||_{F} & \le & \epsilon||Z_{j-1}||_{F} \text{by (\ref{wlp1}) }
\end{eqnarray*}


Inductively, we get desired.

After establishing these properties, we are ready to prove that golfing scheme yields $W^{L}$ that satisfies the desired properties.

1) Proof of condition (1):
\begin{eqnarray*}
||W^{L}|| & = & ||P_{T^{\bot}}(Y_{j_{0}})|| \text{by definition}\\
 & \le & \sum_{j=1}^{j_{0}}\frac{1}{q}||P_{T^{\bot}}P_{\Omega_{j}}Z_{j-1}||\text{ because }Y_{j}=Y_{j-1}+q^{-1}P_{\Omega_{j}}(Z_{j-1})\\
 & = & \sum_{j=1}^{j_{0}}||P_{T^{\bot}}(\frac{1}{q}P_{\Omega_{j}}Z_{j-1}-Z_{j-1})||\text{ because }Z_{j}\in T\\
 & \le & \sum_{j=1}^{j_{0}}||(\frac{1}{q}P_{\Omega_{j}}Z_{j-1}-Z_{j-1})||\text{ because }||P_{T^{\bot}}(M)||\le||M||\\
 & \le & C_{0}^{'}\sqrt{\frac{n\log n}{q}}\sum_{j=1}^{j_{0}}||Z_{j-1}||_{\infty}\text{ because Fact(\ref{fact3})}\\
 & \le & C_{0}^{'}\sqrt{\frac{n\log n}{q}}\sum_{j=1}^{j_{0}}\epsilon^{j}||UV^{*}||_{\infty} \text{by (\ref{wlp2})}\\
 & \le & C_{0}^{'}\sqrt{\frac{n\log n}{q}}\frac{1}{1-\epsilon}||UV^{*}||_{\infty} \text{by bound on geometric series}\\
 & \le & C_{0}^{'}\sqrt{\frac{n\log n}{q}}\frac{1}{1-\epsilon}\frac{\sqrt{\mu r}}{n}\text{by RPCA assumptions}\\
 & \le & C^{''}\epsilon<\frac{1}{4}\text{for some constant }C^{''}
\end{eqnarray*}


2) Proof of condition (2) : First, we expand,
\begin{eqnarray*}
||P_{\Omega}(UV^{*}+W^{L})||_{F} & = & ||P_{\Omega}(UV^{*}+P_{T^{\bot}}Y_{j_{0}})||_{F}
\end{eqnarray*}

Then, because $P_{\Omega}(Y_{j_{0}})=P_{\Omega}(\sum_{j}P_{\Omega_{j}}Z_{j-1})=0$ and $P_{\Omega}(P_{T}(Y_{j_{0}})+P_{T^{\bot}}(Y_{j_{0}}))=0$, we have,
\begin{eqnarray*}
||P_{\Omega}(UV^{*}+W^{L})||_{F} & = & ||P_{\Omega}(UV^{*}-P_{T}Y_{j_{0}})||_{F}
\end{eqnarray*}


Continuing,
\begin{eqnarray*}
||P_{\Omega}(UV^{*}+W^{L})||_{F} & = & ||P_{\Omega}(Z_{j_{0}})||_{F} \text{by definition}\\
 & \le & ||Z_{j_{0}}||_{F} \text{because of summing over larger set} \\
 & \le & \epsilon^{j_{0}}\sqrt{r} \text{by (\ref{wlp3})} \\
 & \le & \sqrt{r}\frac{1}{n^{2}}\le\frac{\lambda}{4} \text{by the choice of} \lambda \text{and} \epsilon
\end{eqnarray*}


3) Proof of condition (3) :
\begin{eqnarray*}
||P_{\Omega^{\bot}}(UV^{*}+W^{L})||_{\infty} & = & ||P_{\Omega^{\bot}}(Z_{j_{0}}+Y_{j_{0}})||_{\infty}\text{by definition}\\
 & \le & ||Z_{j_{0}}||_{\infty}+||Y_{j_{0}}||_{\infty} \text{by triangle inequality and summing over larger set}\\
 & \le & ||Z_{j_{0}}||_{F}+||Y_{j_{0}}||_{\infty} \text{by the properties of Frobenius and infinite norms}\\
 & \le & \frac{\lambda}{8}+||Y_{j_{0}}||_{\infty} \text{similar argument as in Proof of condition (2)}
\end{eqnarray*}


Moreover, we have
\begin{eqnarray*}
||Y_{j_{0}}||_{\infty} & \le & q^{-1}\sum_{j}||P_{\Omega_{j}}Z_{j-1}||_{\infty} \text{ by triangle inequality}\\
 & \le & q^{-1}\sum_{j}||Z_{j-1}||_{\infty} \text{ by summing over a larger set}\\
 & \le & q^{-1}\sum_{j}\epsilon^{j}\frac{\sqrt{\mu r}}{n} \text{ by (\ref{wlp2})}\\
 & \le & \frac{\lambda}{8}\text{ if \ensuremath{\epsilon}\ \text{is sufficiently small}}
\end{eqnarray*}

\end{proof}


%-----------------------------------------------------------------------------------------------------------------------------------------------------------------
\subsection{Proof of the Lemma about least square construction and dual certificate }

Construction of $W^{S}$:

\[
W^{S}=\lambda P_{T^{\bot}}((P_{\Omega}-P_{\Omega}P_{T}P_{\Omega})^{-1}sign(S_{0}))
\]

Now we present the proof of Lemma(\ref{ws}). 
\begin{proof}
We consider the sign of $S_{0}$ to be distrbuted as follows
\[
sign(S_{0})_{i,j}=\begin{cases}
1 & \text{wp }\frac{\rho}{2}\\
0 & \text{wp }1-\rho\\
-1 & \text{wp }\frac{\rho}{2}
\end{cases}
\]


1) Proof of condition (1) :

I) We note the we can separate $W^{S}$into two parts and then bound them separately.

\begin{eqnarray*}
W^{S} & = & \lambda P_{T^{\bot}}(sign(S_{0}))+\lambda P_{T^{\bot}}(\sum_{k\ge1}(P_{\Omega}P_{T}P_{\Omega})^{k}(sign(S_{0})))
\end{eqnarray*}


II) Then, we have
\begin{eqnarray*}
\lambda || P_{T^{\bot}}(sign(S_{0})) || & \le & \lambda||sign(S_{0})|| \text{ by (\ref{property: p2})}\\
 & = & \frac{1}{\sqrt{n}}||sign(S_{0})|| \text{ by the choice of}  \lambda\\
 & \le & 4\sqrt{\rho}\text{with high probability}
\end{eqnarray*}


where the last inequality uses the fact that for the entry-wise distribution of $sign(S_{0}$) , we can have $||sign(S_{0})||\le4\sqrt{n\rho}$ holds with high probability.

III) Now, for the other part, $\lambda P_{T^{\bot}}(\sum_{k\ge1}(P_{\Omega}P_{T}P_{\Omega})^{k}(sign(S_{0})))$, we bound it by first expressing it in the form of $<X,sign(S_{0})>$ and then claim that with high probability, this term is bounded above as desired. Let $R=\sum_{k\ge1}(P_{\Omega}P_{T}P_{\Omega})^{k}$ ,
then we have,
\begin{eqnarray*}
||P_{T^{\bot}}(R(sign(S_{0})))|| & \le & ||R(sign(S_{0}))||\\
 & \le & 4\sup_{x,y\in N}<y,R(sign(S_{0})x)>
\end{eqnarray*}


where the last inequality uses the fact that there exists a $\frac{1}{2}-net$ of the Eucledean ball and it has at most $6^{n}$ elements. Continuing, we have
\begin{eqnarray} 
||P_{T^{\bot}}(R(sign(S_{0})))|| & \le & 4\sup_{x,y\in N}<y,R(sign(S_{0})x)>\\
 & = & 4\sup_{x,y\in N}<yx^{*},R(sign(S_{0}))>\\
 & = & 4\sup_{x,y\in N}<R(yx^{*}),sign(S_{0})>\label{net}
\end{eqnarray}


and that we denote $X(x,y)=<R(yx^{*}),sign(S_{0})>$ afterwards.

Note that, by Hoeffding's inequality, we have,
\begin{eqnarray*}
Pr(|X(x,y)|>t\mid\Omega) & \le & 2exp(-\frac{t^{2}}{2||R(xy^{*})||_{F}^{2}})
\end{eqnarray*}


This gives,
\begin{eqnarray*}
Pr(||P_{T^{\bot}}(R(sign(S_{0})))||>4t\mid\Omega) & \le & Pr(||R(sign(S_{0}))||>4t\mid\Omega)\\
 & \le & Pr(\sup_{x,y}|X(x,y)|>t\mid\Omega) \text{ by (\ref{net})}\\
 & \le & 2N^{2}exp(-\frac{t^{2}}{2||R||_{F}^{2}})\text{because }||yx^{*}||_{F}\le1
\end{eqnarray*}


Now, we proceed to bound the probability without the condition on $\Omega$.

First, note that the event of $||P_{\Omega}P_{T}||\le\sigma=\rho+\epsilon$, implies that $||R||\le(\frac{\sigma^{2}}{1-\sigma^{2}})^{2}$. Thus, unconditionally, we have
\begin{eqnarray*}
Pr(||R(sign(S_{0}))||>4t) & \le & 2|N|^{2}exp(\frac{-t^{2}}{2(\frac{\sigma^{2}}{1-\sigma^{2}})^{2}})+Pr(||P_{\Omega}P_{T}||>\sigma)\\
 & \le & 2\cdot6^{2n}exp(\frac{-t^{2}}{2(\frac{\sigma^{2}}{1-\sigma^{2}})^{2}})+Pr(||P_{\Omega}P_{T}||>\sigma)
\end{eqnarray*}


Thus, where we finally put $t=\frac{1}{16}$
\begin{eqnarray*}
Pr(\lambda||R(sign(S_{0}))||>4t) & \le & 2\cdot6^{2n}exp(\frac{-\frac{t^{2}}{\lambda^{2}}}{2(\frac{\sigma^{2}}{1-\sigma^{2}})^{2}})+Pr(||P_{\Omega}P_{T}||>\sigma)
\end{eqnarray*}


With $\lambda=\sqrt{\frac{1}{n}},$we have this probability$\to0$ as $n\to\infty$. Thus with high probability $||W^{S}||\le\frac{1}{4}$

2) Proof of condition (2) :

The idea is that we first express $P_{\Omega^{\bot}}(W^{S})$ in the form of $<X,sign(S_{0})>$and we can derive upper bound on it if highly probably event of $\{||P_{\Omega}P_{T}||\le\sigma\}$ for some small $\sigma=\rho+\epsilon$ holds .

I) First,
\begin{eqnarray}
P_{\Omega^{\bot}}(W^{S}) & = & P_{\Omega^{\bot}}(\lambda(I-P_{T})(\sum_{k\ge0}(P_{\Omega}P_{T}P_{\Omega})^{k})sign(S_{0})) \text{ by } P_{T^{\bot} = I- P_T}\\
 & = & -\lambda P_{\Omega^{\bot}}P_{T}(P_{\Omega}-P_{\Omega}P_{T}P_{\Omega})^{-1}sign(S_{0}) \text{ by  summing over terms and canceling} \label{wsp1}
\end{eqnarray}


For $(i,j)\in\Omega^{C}$, we have
\begin{eqnarray*}
e_{i}^{*}W^{S}e_{j} & = & <e_{i}e_{j}^{*},W_{S}> \text{ by property of trace}\\
 & = & <e_{i}e_{j}^{*},-\lambda P_{\Omega^{\bot}}P_{T}(P_{\Omega}-P_{\Omega}P_{T}P_{\Omega})^{-1}sign(S_{0})> \text{ by (\ref{wsp1})}\\
 & = & -\lambda<e_{i}e_{j}^{*},P_{T}(P_{\Omega}-P_{\Omega}P_{T}P_{\Omega})^{-1}sign(S_{0})> \text{ by rearranging terms}\\
 & = & -\lambda<e_{i}e_{j}^{*},P_{T}P_{\Omega}(P_{\Omega}-P_{\Omega}P_{T}P_{\Omega})^{-1}sign(S_{0})> \text{ by the property of the inverse }\\
 & = & -\lambda<e_{i}e_{j}^{*},P_{T}\sum_{k\ge0}(P_{\Omega}P_{T}P_{\Omega})^{k}sign(S_{0})> \text{ by infinite sum representation}
\end{eqnarray*}


Noting that $P_{\Omega},P_{T}$ are self-adjoint, thus, we have
\begin{eqnarray}
e_{i}^{*}W^{S}e_{j} & = & \lambda<-(P_{\Omega}-P_{\Omega}P_{T}P_{\Omega})^{-1}P_{\Omega}P_{T}(e_{i}e_{j}^{*}),sign(S_{0})>
\label{wsp2}
\end{eqnarray}


where we now denote $X(i,j)=-(P_{\Omega}-P_{\Omega}P_{T}P_{\Omega})^{-1}P_{\Omega}P_{T}(e_{i}e_{j}^{*})$

II)We now consider, where we put $t=\frac{1}{4}$,
\begin{eqnarray*}
Pr(||P_{\Omega^{\bot}}(W^{S})||_{\infty}>t\lambda\mid\Omega) & \le & \sum_{(i,j)\in\Omega^{C}}Pr(|e_{i}^{*}W^{S}e_{j}|>t\lambda|\Omega) \text{ by union bound}\\
 & \le & n^{2}Pr(|e_{i}^{*}W^{S}e_{j}|>t\lambda|\Omega)\text{ for some (i,j)} \text{ by taking the maximum}\\
 & = & n^{2}Pr(|<X(i,j),sign(S_{0})>|>t|\Omega) \text{ by (\ref{wsp2})}\\
 & \le & 2n^{2}exp(-\frac{2t^{2}}{4||X(i,j)||_{F}})\text{ because of Hoeffding's inequality}
\end{eqnarray*}


III) We then proceed to bound the $||X(i,j)||$. On the event of $\{||P_{\Omega}P_{T}||\le\sigma\}$,
we have ,
\begin{eqnarray*}
||P_{\Omega}P_{T}(e_{i}e_{j}^{*})||_{F} & \le & ||P_{\Omega}P_{T}||\cdot||P_{T}(e_{i}e_{j}^{*})||_{F} \text{ by property of spectral norm}\\
 & \le & \sigma\sqrt{\frac{2\mu r}{n}} \text{ by (\ref{property: p4}) and the bound on } ||P_{\Omega}P_{T}||
\end{eqnarray*}


Moreover, we have
\begin{eqnarray*}
||(P_{\Omega}-P_{\Omega}P_{T}P_{\Omega})^{-1}|| & \le & \sum_{k\ge0}||(P_{\Omega}P_{T}P_{\Omega})^{k}|| \text{ by triangle inequality}\\
 & \le & \frac{1}{1-\sigma} \text{  by the bound on } ||P_{\Omega}P_{T}||
\end{eqnarray*}


Finally, we have
\begin{eqnarray*}
||X(i,j)||_{F} & \le & 2\sigma^{2}\frac{\frac{\mu r}{n}}{(1-\sigma)^{2}}
\end{eqnarray*}


Combining, we have
\begin{eqnarray*}
Pr(||P_{\Omega^{\bot}}W^{S}||>t\lambda) & \le & 2n^{2}exp(\frac{-t^{2}n(1-\sigma)^{2}}{4\sigma^{2}(\mu r)})+Pr(||P_{\Omega}P_{T}||\ge\sigma)\\
 & \le & \epsilon\text{ if }\mu r<\rho_{r}^{'}\frac{n}{\log n}
\end{eqnarray*}

\end{proof}



\subsection{Proof of the equivalence of the Bernoulli sampling and uniform sampling model } \label{sub:bernoullisampling}
To complete the story about the equivalence of sampling model, we present theorem.
\begin{theorem}
Let $E$ be the event that the recovery of $(L_{0},S_{0})$ is exact
through the RPCA. Then, $\forall\epsilon>0$,
\end{theorem}

\begin{itemize}
\item With $\rho=\frac{m}{n^{2}}+\epsilon$, $E$ holds with high probability when the sparse matrix $S_{i,j}\sim Bern(\rho)$ iid $\Longrightarrow$$E$ holds with high probability when the sparse matrix $S\sim Uniform(m)$.
\item With $\rho=\frac{m}{n^{2}}-\epsilon$, $E$ holds with high probability when the sparse matrix $S\sim Uniform(m)$ $\Longrightarrow$ $E$ holds with high probability when the sparse matrix $S_{i,j}\sim Bern(\rho)$
iid
\end{itemize}
\begin{proof}
Let us use the notation of subscrpt to denote the underlying sampling process, e.g. ,$P_{B(\rho)}(E)$ and $P_{U(m)}(E)$ be the probability of success recovery using Bernoulli sampling and uniform sampling respectively. We then upper and lower bound the difference of $P_{B(\rho)}(E)-P_{U(m)}(E)$ and show that the difference goes to zero as the dimension of the matrix $n\to\infty$. \\

\begin{eqnarray*}
 &  & P_{B(\rho)}(E)\\
 & = & \sum_{i=0}^{n^{2}}P_{B(\rho)}(|\Omega|=i)P_{B(\rho)}(E\mid|\Omega|=i)\\
 & = & \sum_{i=0}^{n^{2}}P_{B(\rho)}(|\Omega|=i)P_{U(i)}(E)\\
 & \le & \sum_{i=0}^{m-1}P_{B(\rho)}(|\Omega|=i)+\sum_{i=m}^{n^{2}}P_{U(i)}(E)P_{B(\rho)}(|\Omega|=i)\\
 & \le & \sum_{i=0}^{m-1}P_{B(\rho)}(|\Omega|=i)+\sum_{i=m}^{n^{2}}P_{U(i)}(E)P_{B(\rho)}(|\Omega|=i)\\
 & \le & \sum_{i=0}^{m-1}P_{B(\rho)}(|\Omega|=i)+\sum_{i=m}^{n^{2}}P_{U(m)}(E)P_{B(\rho)}(|\Omega|=i)\\
 & \le & P_{B(\rho)}(|\Omega|<m)+P_{U(m)}(E)
\end{eqnarray*}


This gives, $P_{B(\rho)}(E)-P_{U(m)}(E)\le P_{B(\rho)}(|\Omega|<m)$. With $\rho=\frac{m}{n^{2}}+\epsilon$, by law of large number, when $n\to\infty$ we get, $P_{B(\rho)}(E)\le P_{U(m)}(E)$ .

On the other hand,
\begin{eqnarray*}
 &  & P_{B(\rho)}(E)\\
 & \ge & \sum_{i=0}^{m}P_{B(\rho)}(|\Omega|=i)P_{B(\rho)}(E\mid|\Omega|=i)\\
 & \ge & P_{U(m)}(E)\sum_{i=0}^{m}P_{B(\rho)}(|\Omega|=i)\\
 & = & P_{U(m)}(E)(1-P_{B(\rho)}(|\Omega|>m))\\
 & \ge & P_{U(m)}(E)-P_{B(\rho)}(|\Omega|>m)
\end{eqnarray*}


This gives, $P_{B(\rho)}(E)-P_{U(m)}\ge-P_{B(\rho)}(|\Omega|>m)$. With $\rho=\frac{m}{n^{2}}-\epsilon$, by law of large number, when $n\to\infty$ we get, $P_{B(\rho)}(E)\ge P_{U(m)}(E)$ .
\end{proof}


%-----------------------------------------------------------------------------------------------------------------------------------------------------------------
\subsection{Proof of the form of sub-differential of nuclear norm } \label{sub:nuclearnorm}
To complete the story on the structure of subdifferential of nuclear norm, we present the following justifications.
\begin{definition}
For matrix norms $||\cdot||$ which satisfy $||UAV||=||A||$ $\forall U,V$ being orthonormal, then they are called orthogonally invariant norm.

\begin{definition}
For orthogonally invariant norm $||\cdot||$ which is defined by its singular values $||A||=\phi(\vec{\sigma})$ where $\vec{\sigma}$ are the singular values of $A$, we call the function $\phi$ as a symmetric gauge function if it is a norm and it satisfies $\phi(\vec{\sigma})=\phi(\epsilon_{1}\sigma_{i_{1}},...,\epsilon_{n}\sigma_{i_{n}})$ for any permulation of $(i_{1},...,i_{n})$ of $(1,...,n)$ and $\epsilon_{i}=\pm1$.

\begin{fact}
\label{fact10}
For orthogonally invariant norm $||\cdot||$ with symmetric
gauge function $\phi$, the sub-differential is given by

\begin{eqnarray*}
\partial||A|| & = & \{Udiag(\vec{d})V\mid A=U\Sigma V^{T},\vec{d}\in\partial\phi(\vec{d}),U\in R^{m},V\in R^{n}\}
\end{eqnarray*}

\end{fact}
\end{definition}
\end{definition}

\begin{thm}
Let $A=U^{(1)}\Sigma{V^{(1)}}^{T}$ then $\partial||A||_{*}=\{U^{(1)}{V^{(1)}}^{T}+W:||W||\le1,{U^{(1)}}^{T}W=0,WV^{(1)}=0\}$
\end{thm}

\begin{proof}
We take the symmetric gauge function as $||\cdot||_{1}$and then apply the Fact (\ref{fact10}) and will obtain desired.
\end{proof}

